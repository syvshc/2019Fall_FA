%数学式
\usepackage{mathtools,extarrows,amsfonts,amssymb,bm,mathrsfs} %数学式宏包, 更多箭头, 黑板体等数学字体,leqslant等符号
\usepackage{amsthm} %定理环境
\DeclareMathOperator{\rad}{rad}
\DeclareMathOperator{\diam}{diam}
\newcommand\me{\ensuremath{\mathrm{e}}}
\newcommand\imag{\mathrm{i}}

\newcommand\R{\ensuremath{\mathbb{R}}}
\newcommand\J{\ensuremath{\mathbb{J}}}
\newcommand\Q{\ensuremath{\mathbb{Q}}}
\newcommand\Z{\ensuremath{\mathbb{Z}}}
\newcommand\N{\ensuremath{\mathbb{N}}}
\newcommand\CN{\mathcal{N}}
\newcommand\CB{\mathcal{B}}
\renewcommand\Re{\mathrm{Re\,}}
\renewcommand\Im{\mathrm{Im\,}}
\newcommand{\dint}{\displaystyle\int}
\newcommand{\diff}{\,\mathrm{d}}
\usepackage{tasks}
\NewTasks[counter-format=(tsk[1]), item-indent=2em, label-offset=1em]{lpbn}
\NewTasks[counter-format=(tsk[a]), item-indent=2em, label-offset=1em]{alpbn}
\NewTasks[counter-format=tsk[A].]{xrze}[*]

%版式
\usepackage[a4paper,left=2.5cm,right=2.5cm,top=2.5cm,bottom=2cm]{geometry} %边距
\setlength{\headheight}{13pt}
\usepackage{fancyhdr} % 页眉页脚
\pagestyle{fancy}
\fancyhf{}
\fancyhead[OL]{\fangsong \rightmark}
\fancyhead[ER]{\fangsong \leftmark}
\fancyhead[OR,EL]{\thepage}

%辅助
\usepackage{array,diagbox,booktabs,tabularx} %数组环境, 表格中可以添加对角线, 可以调整表格中线的宽度, 可以控制表格宽度并使其自动换行
%\usepackage{enumerate} %列表环境
\usepackage[shortlabels]{enumitem} % 继承并扩展了enumerate宏包的功能
\setlist[1]{left=\parindent..2\parindent}
\setlist{noitemsep}



%交叉引用
\usepackage{nameref}
\usepackage{prettyref}
\usepackage[colorlinks, linkcolor=Sumire]{hyperref}
\usepackage{graphicx}
\renewcommand\C{\ensuremath{\mathbb{C}}}
%浮动体
\usepackage{caption} %使用浮动体标题
\usepackage{subfig} %子浮动体

% 新定义计数器
\newcounter{FA}[section]
\renewcommand{\theFA}{\thesection.\arabic{FA}}

%新定义定理环境类型
\newtheoremstyle{normal}% name
{3pt}% Space above
{3pt}% Space below
{}% Body font
{2\ccwd}% Indent amount
{\bfseries}% Theorem head font
{}% Punctuation after theorem head
{1\ccwd}% Space after theorem head
{}% Theorem head spec (can be left empty, meaning `normal' )
\newtheoremstyle{Thm}% hnamei
{3pt}% Space above
{3pt}% Space below
{\kaishu}% Body font
{2\ccwd}% Indent amount
{\bfseries}% Theorem head font
{}% Punctuation after theorem head
{1\ccwd}% Space after theorem head
{}% Theorem head spec (can be left empty, meaning `normal' )

%新定义定理环境
\theoremstyle{normal}
%\theoremstyle{Thm}
\newtheorem{Thm}[FA]{定理}
\newtheorem{Thmn}{定理}
\renewcommand{\theThmn}{\theFA$'$}

\newtheorem{Lemma}[FA]{引理}
%\newtheorem{Exp}{例}[subsection]
\newtheorem*{Prf}{证明}
\newtheorem{Rmk}[FA]{注}
\newtheorem*{Solve}{解}
\newtheorem{Def}[FA]{定义}
\newtheorem{Defn}{定义}
\renewcommand{\theDefn}{\thefA$'$}
\newtheorem{Cor}[FA]{推论}
\newtheorem{Ex}[FA]{例}
\newtheorem{Prop}[FA]{命题}


%新定义命令
\newcommand{\abs}[1]{\ensuremath{\left| \,#1\, \right| }}
\newcommand{\lrangle}[1]{\langle #1 \rangle}
\newcommand{\degree}{\ensuremath{^{\circ}}}
\newcommand{\bs}{\backslash}
\newcommand{\baro}{\overline}
%\renewcommand{\labelenumi}{(\theenumi)}