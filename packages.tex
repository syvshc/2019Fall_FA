%数学式
\usepackage{mathtools,extarrows,amsfonts,amssymb,bm,mathrsfs} %数学式宏包, 更多箭头, 黑板体等数学字体,leqslant等符号
\usepackage{amsthm} %定理环境
\DeclareMathOperator{\rad}{rad}
%\DeclareMathOperator{\mi}{i}
%\DeclareMathOperator{\me}{e}
\newcommand\me{\mathrm{e}}
\newcommand\imag{\mathrm{i}}
\newcommand\R{\mathbb{R}}
\renewcommand\Re{\mathrm{Re\,}}
\renewcommand\Im{\mathrm{Im\,}}
\newcommand{\Z}{\mathbb{Z}}
\newcommand{\N}{\mathbb{N}}
\newcommand{\dint}[2]{{\displaystyle\int_{#1}^{#2}}}
\newcommand{\diff}{\,\mathrm{d}}
\usepackage{tasks}
\NewTasks[counter-format=(tsk[1]), item-indent=2em, label-offset=1em]{lpbn}
\NewTasks[counter-format=(tsk[a]), item-indent=2em, label-offset=1em]{alpbn}
\NewTasks[counter-format=tsk[A].]{xrze}[*]

%版式
\usepackage[a4paper,left=2.5cm,right=2.5cm,top=2.5cm,bottom=2cm]{geometry} %边距
\setlength{\headheight}{13pt}
\usepackage{fancyhdr} % 页眉页脚
\pagestyle{fancy}
\fancyhf{}
\fancyhead[L]{\fangsong \leftmark}
\fancyhead[C]{\fangsong \rightmark}
\fancyhead[R]{\thepage}

%辅助
\usepackage{array,diagbox,booktabs,tabularx} %数组环境, 表格中可以添加对角线, 可以调整表格中线的宽度, 可以控制表格宽度并使其自动换行
\usepackage{enumerate} %列表环境



%交叉引用
\usepackage{nameref}
\usepackage{prettyref}
\usepackage{hyperref}
\usepackage{graphicx}

%浮动体
\usepackage{caption} %使用浮动体标题
\usepackage{subfig} %子浮动体

%新定义定理环境类型
\newtheoremstyle{normal}% name
{3pt}% Space above
{3pt}% Space below
{}% Body font
{2\ccwd}% Indent amount
{\bfseries}% Theorem head font
{}% Punctuation after theorem head
{1\ccwd}% Space after theorem head
{}% Theorem head spec (can be left empty, meaning `normal' )
\newtheoremstyle{Thm}% hnamei
{3pt}% Space above
{3pt}% Space below
{\itshape}% Body font
{2\ccwd}% Indent amount
{\bfseries}% Theorem head font
{}% Punctuation after theorem head
{1\ccwd}% Space after theorem head
{}% Theorem head spec (can be left empty, meaning `normal' )

%新定义定理环境
\theoremstyle{Thm}
\newtheorem{Thm}{Theorem}
\newtheorem{Thmn}{Theorem}
\renewcommand{\theThmn}{\theThm'}
\theoremstyle{normal}

\newtheorem*{Rmk}{Remark}

\newtheorem{Def}{Definition}
\newtheorem{Defn}{Definition}
\renewcommand{\theDefn}{\theDef$^\prime$}
\newtheorem{Cor}{Corallary}
\newcommand{\degree}{\ensuremath{^{\circ}}}


%新定义命令
\newcommand{\abs}[1]{\ensuremath{\left| \,#1\, \right| }}
\newcommand{\lrangle}[1]{\langle #1 \rangle}