% !TeX root = main.tex

\chapter{线性算子与线性泛函}

\section{连续线性映射}

	\begin{Definition}\index{X!线性算子}
	设$ E, F $是$ \K $上的线性空间, 若映射$ u : E\to F $满足
	\[
	\forall \lambda\in\K\,\forall x,y\in E\,(u(\lambda x+y)=\lambda u(x)+u(y)),
	\]
	则称$ u : E\to F $是一个\textbf{线性映射}或\textbf{线性算子}, 并记$ \mathcal L(E,F) $是$ E $到$ F $线性算子的全体.
	\end{Definition}
	
	特别地, 当$ E=F $时, 用$ \mathcal L(E) $记$ \mathcal{L}(E,E) $.
	
	\begin{Example}~
	\begin{enumerate}[(1)]
	\item $ \forall \lambda\in\K $, 定义
	\[
	M_\lambda : E\to E,\qquad x\mapsto\lambda x,
	\]
	则$ M_\lambda\in\mathcal L(E) $. 而对$ y\in E $, 定义
	\[
	T_y : E\to E,\qquad x\mapsto x+y,
	\]
	则$ T_y\notin\mathcal L(E) $.
	
	\item 设$ E=C[0,1] $, 对$ f\in C[0,1] $, 定义$ M_fg=fg $, 则$ M_f $是线性算子. 再取$ x\in[0,1] $, 定义$ \delta_x(f)=f(x) $, 则$ \delta_x $是线性算子, 再定义
	\[
	I(f)(x)=\int_0^xf(t)\diff t,
	\]
	则$ I $也是线性算子.
	\end{enumerate}
	\end{Example}
	
	\begin{Remark}
	关于$ \mathcal L(E,F) $的一些注记:
	\begin{enumerate}[(1)]
	\item $ \mathcal L(E,F) $本身也是一个线性空间, 其上的线性运算可以定义为
	\[
	\lambda u : E\to F,\qquad x\mapsto\lambda u(x),\\
	u+v : E\to F,\qquad x\mapsto u(x)+v(x),
	\]
	其中$ \lambda\in\K $, $ u, v\in\mathcal L(E,F) $.
	
	\item 对$ u\in\mathcal L(E,F) $, 定义其\textbf{核空间}\index{H!核空间}为
	\[
	\ker u=\{ x\in E : u(x)=0 \}.
	\]
	则$ \ker u $是$ E $的线性子空间, 且$ u $是单射当且仅当$ \ker u=\{0\} $.
	\end{enumerate}
	\end{Remark}
	
	\begin{Theorem}[连续性]\label{thm:线性算子连续性}
	设$ E, F $是$ \K $上的两个赋范空间, 且$ u\in\mathcal L(E,F) $, 则下列命题等价:
	\begin{enumerate}[(1)]
	\item $ u $在$ E $上连续;
	\item $ u $在$ E $上的某一点连续;
	\item $ u $在原点连续;
	\item $ \exists C\geqslant 0\,\forall x\in E\,(\norm{u(x)}\leqslant C\norm{x}) $.
	\end{enumerate}
	\end{Theorem}
	\begin{Proof}
	(1) $ \Rightarrow $ (2) : 显然.
	
	(2) $ \Rightarrow $ (3) : 设$ u $在$ x_0\in E $连续, 那么
	\[
	\forall\varepsilon>0\,\exists r>0\,(x\in B(x_0,r)\Rightarrow\norm{u(x)-u(x_0)}<\varepsilon),
	\]
	则对$ \forall y\in B(0,r) $, 有
	\[
	\norm{u(y)-u(0)}=\norm{u(y)}=\norm{u(x_0+y)-u(x_0)}<\varepsilon,
	\]
	即$ u $在原点连续.
	
	(3) $ \Rightarrow $ (4) : 由$ u $在原点连续可知
	\[
	\exists r_0>0\,(y\in\overline{B(0,r)}\Rightarrow\norm{u(y)}<1).
	\]
	则$ \forall x\in E $, 由
	\[
	\norm{u(x)}=\norm{u\left(\dfrac{xr}{\norm{x}}\right)}\frac{\norm{x}}{r}<\frac{\norm{x}}{r},
	\]
	取$ C=\frac{1}{r} $即可.
	
	(4) $ \Rightarrow $ (1) : 由$ u\in\mathcal{L}(E,F) $, 有
	\[
	\forall x,y\in E\,(\norm{u(x)-u(y)}\leqslant C\norm{x-y}),
	\]
	即$ u $是Lipschitz的, 从而$ u $在$ E $连续.\qed
	\end{Proof}
	
	\begin{Remark}
	上述定理表明在赋范空间上, 连续线性映射都是Lipschitz的, 从而连续线性映射一定一致连续.
	\end{Remark}
	
	\begin{Definition}[算子范数]\index{Y!有界线性算子}
	设$ E,F $都是赋范空间, $ u\in\mathcal{L}(E,F) $. 若
	\[
	\forall C\geqslant 0\,\forall x\in E\,(\norm{u(x)}\leqslant C\norm{x}),
	\]
	则称$ u $是\textbf{有界}的. 再令
	\[
	\norm{u}=\sup_{\norm{x}\ne 0}\frac{\norm{u(x)}}{\norm{x}}
	\]
	称$ \norm{u} $是算子$ u $的\textbf{范数}. 并记$ \mathcal B(E,F) $是$ E $到$ F $的有界线性算子的全体, 且类似地记$ \mathcal{B}(E)=\mathcal{B}(E,E) $.
	\end{Definition}
	
	\begin{Example}
	设$ E=\C^n $, 则$ \mathcal{L}(E)=\mathbb{M}_n(\C)=\mathcal{B}(E) $, 其中$ \mathbb{M}_n(\C) $是$ n $阶复矩阵的全体. 对$ A\in\mathbb{M}_n(\C) $, 定义
	\[
	\norm{A}=\sup_{\norm{x}\ne 0}\frac{\norm{Ax}}{\norm{x}}
	\]
	为$ A $的范数. 则由
	\[
	\norm{A}^2=\sup_{\norm{x}\ne 0}\frac{\norm{Ax}^2}{\norm{x}^2}=\sup_{\norm{x}\ne 0}\frac{\lrangle{Ax,Ax}}{\norm{x}^2}=\sup_{\norm{x}\ne 0}\frac{\lrangle{A^\ast Ax,x}}{\norm{x}^2}=\sup\{\abs{\lambda}\},
	\]
	可知$ A $是有界线性算子. 其中$ \lambda $是$ A^\ast A $的特征值.
	\end{Example}
	
	\begin{Remark}~
	\begin{enumerate}[(1)]
	\item $ \mathcal B(E,F) $是$ \mathcal L(E,F) $的线性子空间, 这由
	\[
	\norm{\lambda u}=\abs{\lambda}\norm{u},\qquad \norm{u+v}\leqslant\norm{u}+\norm{v}
	\]
	可知. 且$ (\mathcal{B}(E,F),\norm{\cdot}) $还是一个赋范空间. 由定理\ref{thm:线性算子连续性}可知$ u $连续等价于$ u $有界.
	
	\item $ u $的有界性实际上指$ \norm{u(x)} $在$ E $的有界集上有界. 实际上除非$ u\equiv 0 $, 否则$ \norm{u(x)} $不可能在全空间上有界.
	
	\item 范数$ \norm{u} $也可以等价地定义为
	\[
	\norm{u}:=\sup_{\norm{x}\ne 0}\frac{\norm{u(x)}}{\norm{x}}=\sup_{\norm{x}\leqslant 1}\norm{u(x)}=\sup_{\norm{x}=1}\norm{u(x)},
	\]
	且$ \norm{u} $就是定义中满足条件的$ C $中最小的.
	\end{enumerate}
	\end{Remark}