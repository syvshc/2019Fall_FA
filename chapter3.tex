% !TeX root = main.tex

\chapter{Banach空间的对偶理论}
\section{半范数}
	\begin{Definition}[半范数]\label{def:半范数}\index{B!半范数}
		设 $ E $ 是数域\K 上的线性空间, 托 $ E $ 上的泛函 $ p : E\to \R $ 满足
		\begin{enumerate}[(1)]
			\item 非负性: $ \forall x\in E\,(p(x)\geqslant0) $;
			\item 齐次性: $ \forall \lambda\in\K\,(p(\lambda x)=\abs{\lambda}p(x)) $;
			\item 三角不等式: $ \forall x, y\in E\,(p(x+y)\leqslant p(x)+p(y)) $.
		\end{enumerate}
		则称 $ p $ 是 $ E $ 的一个\textbf{半范数}. 
	\end{Definition}
	例如: $ E $ 上的范数 $ \norm{\cdot} $ 是一个半范数, 且对 $ f\in \Star{E} $,  $ \abs{f} $ 是一个半范数.
	\begin{Definition}[平衡, 吸收]\label{def:平衡吸收}
		设 $ E $ 是数域 \K 上的线性空间,  $ A\subset E $, 则
		\begin{enumerate}[(1)]
			\item 若 $ \forall\abs{\lambda}\leqslant1\,\forall x\in A\,(\lambda x\in A) $, 则称 $ A $ 是\textbf{平衡}的;\index{P!平衡}
			\item 若 $ \forall x\in E\,\exists \alpha>0\,(\abs{\lambda}\leqslant\alpha\Rightarrow\lambda x \in A) $, 则称 $ A $ 是\textbf{吸收}的. \index{X!吸收}
			\item 若 $ x\in A\Rightarrow -x\in A $, 则称 $ A $ 是\textbf{对称}的. \index{D!对称} 
		\end{enumerate}
	\end{Definition}
	可以从图~\ref{fig:平衡吸收}~来理解平衡和吸收的概念
	\begin{figure}[hp!]
		\begin{center}	
		\begin{tikzpicture}
			\draw[-latex] (-1,0) -- (1,0) node[below] {$ \scriptstyle{x} $};
			\draw[-latex] (0,-1) -- (0,1) node[right] {$ \scriptstyle{y} $};
			\draw (-1,-0.5) -- (1,0.5); 
			\node (A1) at (0,-1.3) {\zihao{6} 平衡, 不吸收};
			\node (B1) at (0.75,-0.75) {$ \mathbb{R}^{2} $};
			\filldraw[fill=gray!30] (4,0) circle[radius=0.75];
			\draw[-latex] (3,0) -- (5,0) node[below] {$ \scriptstyle{x} $};
			\draw[-latex] (4,-1) -- (4,1) node[right] {$ \scriptstyle{y} $};
			\node (A2) at (4,-1.3) {\zihao{6} 平衡, 吸收};
			\node (B2) at (4.8,-0.8) {$ \mathbb{C} $};
			\filldraw[fill=gray!30] (7.25,-0.75) rectangle (8.75,0.75); 
			\draw[-latex] (7,0) -- (9,0) node[below] {$ \scriptstyle{x} $};
			\draw[-latex] (8,-1) -- (8,1) node[right] {$ \scriptstyle{y} $};
			\node (A3) at (8,-1.3) {\zihao{6} 平衡, 吸收, 凸集};
			\node (B3) at (9,-0.8) {$ \mathbb{R}^{2} $};
			\filldraw[fill=gray!30] (11.5,-0.5) -- (12.25,0.75) -- ++(0,-0.5) -- ++(0.5,0) -- cycle;
			\draw[-latex] (11,0) -- ++(2,0) node[below] {$ \scriptstyle{x} $};
			\draw[-latex] (12,-1) -- ++(0,2) node[right] {$ \scriptstyle{y} $};
			\node (A4) at (12,-1.3)  {\zihao{6} 不平衡, 吸收} ;
			\node (B4) at (12.75,-0.75) {$ \mathbb{R}^{2} $};
		\end{tikzpicture}
		\caption{平衡, 吸收图例}\label{fig:平衡吸收}
		\end{center}
	\end{figure}


	\begin{Definition}[Minkowski泛函]\label{def:Minkowski泛函}\index{M!Minkowski泛函}
		设 $ E $ 是线性空间,  $ \varOmega $ 是 $ E $ 中平衡吸收的凸集, 则称
		\[
			P_{\varOmega} : E\to\R\quad x\mapsto \inf\left\{ \lambda>0:\frac{x}{\lambda}\in\varOmega \right\}
		\]
		为对应于 $ \varOmega $ 的 \textbf{Minkowski 泛函}.
	\end{Definition}
	\begin{Theorem}\label{thm:M凸}
		设 $ E $ 是线性空间,  $ \varOmega, \varOmega_{1}, \varOmega_{2} $ 是平衡吸收的凸集, 则:
		\begin{enumerate}[(1)]
			\item $ p_{\varOmega} $ 是 $ E $ 上的半范数, 且 $ \varOmega\subset\left\{ x\in E : p_{\varOmega}(x)\leqslant1 \right\} $;
			\item 若 $ \varOmega_{1}\subset\varOmega_{2} $, 则 $ p_{\varOmega_{1}}\geqslant p_{\varOmega_{2}} $;
			\item 设 $ \varOmega_{3}=\varOmega_{1}\cap \varOmega_{2} $, 则 $ p_{\varOmega_{3}}\geqslant\max\left\{ p_{\varOmega_{1}}, p_{\varOmega_{2}} \right\} $.
		\end{enumerate}
	\end{Theorem}
	\begin{Proof}
		(1) 取 $ I(x)=\left\{ \lambda>0 : \frac{x}{\lambda}\in\varOmega \right\} $. 若 $ \lambda\in I(x) $, 则 $ \forall \mu>\lambda $, 有 $ \frac{x}{\mu}=\frac{x}{\lambda}\cdot\frac{\lambda}{\mu}\in\varOmega $, 故 $ \mu\in I(x) $, 即 $ I(x) $ 为一个区间, 而 $ \forall x\in\varOmega $, 有 $ 1\in I(x) $, 故 $ p_{\varOmega}(x)=\inf I(x)\leqslant1 $, 也即
		\[
			\varOmega\subset\left\{ x\in E : p_{\varOmega}(x)\leqslant1 \right\}.
		\]
		还需说明 $ p_{\varOmega} $ 是半范数. 由 $ p_{\varOmega} $ 定义知 $ p_{\varOmega}\geqslant0 $, 即非负性成立. 而对 $ \alpha\ne 0 $, 有
		\[
			I(\alpha x)=\left\{ \lambda>0 : \frac{\alpha x}{\lambda}\in \varOmega \right\}=\left\{ \lambda>0 : \frac{\alpha x/\abs{\alpha}}{\lambda/\abs{\alpha}} \right\},
		\]
		因为 $ \alpha x/\abs{\alpha}\in \varOmega $ 与 $ x\in\varOmega $ 等价 (由平衡性可知), 从而
		\[
			I(\alpha x)=\left\{ \lambda>0 : \frac{\alpha x/\abs{\alpha}}{\lambda/\abs{\alpha}} \right\}=\left\{ \abs{\alpha}\lambda'>0 : \frac{x}{\lambda'}\in\varOmega \right\} = \abs{\alpha} I(x),
		\]
		两侧同取下确界, 有 $ p_{\varOmega}(\alpha x)=\abs{\alpha}p_{\varOmega}(x) $, 即齐次性成立.

		再证三角不等式. 取 $ \lambda\in I(x), \mu\in I(y) $. 则有 $ x/\lambda\in\varOmega $ 且 $ y/\mu\in\varOmega $. 因为
		\[
			\frac{x+y}{\lambda+\mu} = \frac{x}{\lambda}\cdot\frac{\lambda}{\lambda+\mu}+\frac{y}{\mu}\cdot\frac{\mu}{\lambda+\mu}.
		\]
		因为 $ \abs{\frac{\lambda}{\lambda+\mu}}\leqslant1 $, $ \abs{\frac{\mu}{\lambda+\mu}}\leqslant1 $, 且 $ \frac{\lambda}{\lambda+\mu}+\frac{\mu}{\lambda+\mu}=1 $, 由 $ \varOmega $ 的凸性知 $ \frac{x+y}{\lambda+\mu}\in\varOmega $, 即 $ \lambda+\mu\in I(x+y) $. 也即 $ p_{\varOmega}(x+y)\leqslant\lambda+\mu $, 再由 $ \lambda, \mu $ 的任意性
		\[
			p_{\varOmega}(x+y)\leqslant p_{\varOmega}(x)+p_{\varOmega}(y),
		\]
		即三角不等式成立.

		(2) 若 $ \varOmega\subset\varOmega $, 则由
		\[
			I_{\varOmega_{1}}=\left\{ \lambda>0:\frac{x}{\lambda}\in\varOmega_{1} \right\}\subset\left\{ \lambda>0:\frac{x}{\lambda}\in\varOmega_{2} \right\} = I_{\varOmega_{2}}(x)
		\]
		可知 $ p_{\varOmega_{1}}\geqslant p_{\varOmega_{2}} $. 

		(3) 因为 $ \varOmega_{3}\subset\varOmega_{1}, \varOmega_{3}\subset\varOmega_{2} $. 由~(2)~知 $ p_{\varOmega_{3}}\geqslant p_{\varOmega_{1}} $, $ p_{\varOmega_{3}}\geqslant p_{\varOmega_{2}} $, 故 $ p_{\varOmega_{3}}\geqslant\max\left\{ p_{\varOmega_{1}}, p_{\varOmega_{2}} \right\} $.\qed
	\end{Proof}

	\begin{Proposition}
		设 $ E $ 是赋范空间, $ \varOmega $ 是平衡吸收的闭凸集, 则
		\begin{enumerate}[(1)]
			\item $ \varOmega = \{ x\in E : p_{\varOmega}(x)\leqslant 1 \} $;
			\item 若 $ \varOmega $ 有界, 则 $ p_{\varOmega}(x)=0 $ 的充分必要条件是 $ x=0 $;
			\item 若 $ 0 $ 是 $ \varOmega $ 的内点, 则 $ p_{\varOmega} $ 一致连续. 
		\end{enumerate}
	\end{Proposition}
	\begin{Proof}
		(1) 由定理~\ref{thm:M凸}~, 有 $ \varOmega\subset\{ x\in E : p_{\varOmega}(x)\leqslant1 \} $. 而
		\[
			\forall n\geqslant 1\,\left( \frac{x}{1+1/n}\in\varOmega \right)
		\]
		令 $ n\to \infty $, 有 $ x\in\varOmega $, 于是 $ \varOmega = \{ x\in E : p_{\varOmega}(x)\leqslant 1 \} $.

		(2) 因为 $ \varOmega $ 有界, 则存在 $ r>0 $ 使得 $ \varOmega\subset B(0, r) $, 从而对任意 $ x $, 有 $ \frac{rx}{\norm{x}}\notin\varOmega $. 于是
		\[
			p_{\varOmega}(x)\geqslant\frac{\norm{x}}{r},
		\]
		从而 $ p_{\varOmega}(x)=0\Longleftrightarrow\norm{x}=0\Longleftrightarrow x=0 $.

		(3) $ 0 $ 是 $ \varOmega $ 的内点, 即存在 $ r>0 $ 使得 $ B(0, r)\subset\varOmega $, 则对任意 $ x $, 都有 $ \frac{rx}{2\norm{x}}\in \varOmega $, 于是
		\[
			p_{\varOmega}(x)\leqslant\frac{2\norm{x}}{r},
		\]
		从而
		\[
			\abs{p_{\varOmega}(x)-p_{\varOmega}(y)}\leqslant\max\{ p_{\varOmega}(x-y), p_{\varOmega}(y-x) \}\leqslant\frac{2\norm{x-y}}{r}
		\]
		即 $ p_{\varOmega} $ 是 Lipschitz 的, 从而一致连续. (因为 $ \varOmega $ 是平衡的, 所以实际上 $ p_{\varOmega}(x-y) = p_{\varOmega}(y-x) $.)\qed
	\end{Proof}
	\begin{Remark}
		若 $ \varOmega $ 是平衡吸收的开凸集, 则 $ \varOmega=\{ x\in E:p_{\varOmega}(x)<1 \} $. 若 $ \varOmega $ 只是吸收的凸集, 仍可定义 Minkowski 泛函 $ p_{\varOmega} $, 但此时 $ p_{\varOmega} $ 不是半范数, 但 $ p_{\varOmega} $ 仍满足
		\begin{enumerate}[(1)]
			\item $ \forall t>0\,(p_{\varOmega}(tx)=tp_{\varOmega}(x)) $;
			\item $ \forall x, y\in\varOmega\,(p_{\varOmega}(x+y)\leqslant p_{\varOmega}(x)+p_{\varOmega}(y)) $;
			\item 设 $ E $ 是赋范空间,  $ \varOmega $ 是开集, 则 $ \{ x\in E:p_{\varOmega}(x)<1 \}=\varOmega $, 即 $ \forall x\notin\varOmega, p_{\varOmega}(x)\geqslant1 $.
		\end{enumerate}
	\end{Remark}
	推广上述 Minkowski 泛函, 有
	\begin{Definition}[次线性泛函]\label{def:次线性泛函}\index{C!次线性泛函}
		设 $ E $ 是\R 上的线性空间, 若泛函 $ p:E\to \R $ 满足
		\begin{enumerate}[(1)]
			\item 齐次性: $ \forall t>0\,\forall x\in E\,(p(tx)=tp(x)) $;
			\item 次加性: $ \forall x, y\in E\,(p(x+y)\leqslant p(x)+p(y)) $.
		\end{enumerate}
		则称泛函 $ p $ 是 $ E $ 上的\textbf{次线性泛函}. 
	\end{Definition}
	例如: $ E $ 上的线性泛函是次线性泛函; Minkowski 泛函是次线性泛函; 半范数也是次线性泛函. 

\section{Hahn-Banach定理}
\subsection{分析形式的 Hahn-Banach定理}
	\begin{Proposition}
		设 $ E $ 是线性空间,  $ \rho $ 是 $ E $ 上 (推广的) 线性泛函,  $ \rho : E\to \K $, 记 $ F= \ker \rho\subset E $ 则 $ E/ F $ 是线性空间, 且 $ \codim F = \dim E/F=1 $.
	\end{Proposition}
	\begin{Proof}
		若 $ x_{1}+F, x_{2}+F $ 在 $ E/F $ 中线性无关, 则 $ x_{1}+F\ne F, x_{2}+F\ne F $, 即 $ \rho(x_{1})\ne 0, \rho(x_{2})\ne 0 $, 由 $ \rho $ 的线性性可知
		\[
			\rho(x_{2}\rho(x_{1})-x_{1}\rho(x_{2}))=\rho(x_{2})\rho(x_{1})-\rho(x_{1})\rho(x_{2})=0
		\]
		即 $ x_{2}\rho(x_{1})-x_{1}\rho(x_{2})+F=F $, 也即 $ \rho(x_{1})(x_{2}+F)-\rho(x_{2})(x_{1}+F)=F $. 矛盾.\qed
	\end{Proof}
	上一命题说明了对一般的线性空间 $ E $, 余维数为 1 可用
	\[
		\exists x\in E\sm F \,(E=F+\K x)
	\]
	来刻画.

	\begin{Lemma}
		设 $ E $ 是线性空间,  $ F $ 是 $ E $ 的线性子空间且 $ \codim F=1 $, 并设 $ p:E\to \R $ 是次线性泛函,  $ f:F\to \R $ 是线性泛函, 在 $ F $ 上成立 $ f\leqslant p $, 则存在线性泛函 $ \tilde{f}:E\to\R $ 使得
		\[
			(\tilde{f}|_{F}=f)\land(\forall x\in E\,(\tilde{f}(x)\leqslant p(x))).
		\]
	\end{Lemma}
	\begin{Proof}
		由 $ \codim F =1 $ 可知 $ \exists x_{0}\in E\sm F $ 使得 $ E = \R x_{0}+F $. 设 $ \tilde{f}:E\to\R $ 是 $ f $ 的线性延拓, 即 $ \tilde{f} $ 线性且 $ \tilde{f}|_{F}=f $. 则
		\[
			\forall\lambda\in\R\,\forall x\in F\,(\tilde{f}(\lambda x_{0}+x)=\lambda\tilde{f}(x_{0})+f(x)),
		\] 
		即 $ \tilde{f} $ 由其在 $ x_{0} $ 处的取值唯一确定, 不妨设 $ \tilde{f}(x_{0})=a $, 则
		\[
			\begin{aligned}
				\forall x'\in E\,(\tilde{f}(x')\leqslant p(x')) & \Longleftrightarrow \forall\lambda\in \R\,\forall x\in F\,(\tilde{f}(\lambda x_{0}+x)\leqslant p(\lambda x_{0}+x)) \\
				& \Longleftrightarrow \forall\lambda\in \R\,\forall x\in F\,(\lambda a+f(x)\leqslant p(\lambda x_{0}+x)).
			\end{aligned}
		\]
		从而上式等价于 $ \forall\lambda\in\R $, $ \forall x\in F $.
		\[
			\begin{cases}
				a\leqslant \frac{1}{\lambda}p(\lambda x_{0}+x)-\frac{1}{\lambda}f(x) & ,\lambda\geqslant0\\
				a\geqslant \frac{1}{\lambda}p(\lambda x_{0}+x)-\frac{1}{\lambda}f(x) & ,\lambda<0
			\end{cases}\Longleftrightarrow
			\begin{cases}
				a\leqslant p\left(x_{0}+\frac{x}{\lambda}\right)-f\left( \frac{x}{\lambda} \right) & ,\lambda\geqslant 0\\
				a\geqslant-p\left( -x_{0}-\frac{x}{\lambda} \right) + f\left( -\frac{x}{\lambda} \right) & ,\lambda>0
			\end{cases}
		\]
		注意到上式对任意的 $ x\in F $ 成立. 故可将 $ -x/\lambda $ 替换为 $ x $, 将 $ x/\lambda $ 替换为 $ y $. 则上式化为
		\[
			\forall x, y\in F\,(f(x)-p(-x_{0}+x)\leqslant a\leqslant p(x_{0}+y)-f(y)),
		\]
		故 $ a $ 的存在性等价于
		\[
			f(x)-p(x_{0}+x)\leqslant p(x_{0}+y)-f(y),
		\]
		也即
		\[
			f(x)+f(y)\leqslant p(x_{0}+y)+p(-x_{0}+x),
		\]
		而注意到
		\[
			f(x)+f(y)=f(x+y)\leqslant p(x+y)\leqslant p(x-x_{0})+p(x_{0}+y),
		\]
		故存在 $ a $ 满足 $ a\in\Big[\sup\limits_{x\in F}(f(x)-p(-x_{0}+x)), \inf\limits_{y\in F}(p(x_{0}+y)-f(y))\Big] $\qed
	\end{Proof}
	本节称上一定理为延拓定理。 则由延拓定理, $ \codim F<\infty $ 时重复使用这一定理即可. $ \codim F=\aleph_{0} $ 时由归纳法也有相应命题成立, 而对 $ \codim F > \aleph_{0} $ 的情形则不可避免地用到 Zorn 引理.

	\begin{Theorem}[Hahn-Banach 定理: 实情形]
		设 $ E $ 是实线性空间, $ F $ 是 $ E $ 的线性子空间, $ p: E\to\R $ 是次线性泛函, $ f : F\to\R $ 为线性泛函, 且在 $ F $ 上成立 $ f\leqslant p $, 则存在 $ f $ 的线性延拓 $ \tilde{f}:E\to\R $ 使得
		\[
			(\tilde{f}|_{F}=f)\land(\forall x\in E\,(\tilde{f}(x)\leqslant p(x))).
		\]
	\end{Theorem}
	\begin{Proof}
		设 $ \CF $ 是二元组 $ (G, g) $ 的全体, 其中 $ (G, g) $ 满足
		\begin{enumerate}[(1)]
			\item $ G $ 是 $ E $ 的线性子空间且 $ F\subset G $;
			\item $ g:G\to \R $ 是线性泛函, $ g|_{F}=f $ 且在 $ G $ 上成立 $ g\leqslant p $.
		\end{enumerate}
		并定义 $ \CF $ 上的偏序:
		\[
			(G, g)\leqslant (H, h)\Longleftrightarrow G\subset H \land h|_{G}=g.
		\]
		则 $ (\CF, \leqslant) $ 为一个偏序集. 任取 $ \CG $ 为 $ \CF $ 的全序子集, 并令 $ H = \bigcup_{(G, g)\in\CG}G $, 定义 $ h:H\to\R $ 使得
		\[
			\forall (G, g)\in\CG\,\forall x\in G\,(h(x)=g(x)).
		\]
		则 $ (H, h) $ 是 $ \CG $ 的上界 (此因对 $ \forall (G, g)\in\CG $ 都有 $ h|_{G}=g $ 且 $ h\leqslant p $). 由 Zorn 引理, $ \CF $ 有极大元 $ (M, m) $.

		往证 $ M=E $. 否则 $ \exists x_{0}\in E\sm M $, 令 $ G = \Span\{ M, x_{0} \}=M+\R x_{0} $, 此时 $ \codim_{G}M=1 $, 由延拓定理可知存在 $ g:G\to\R $, 使得
		\[
			(g|_{M} = m) \land (g\leqslant p).
		\]
		即 $ (M, m)\leqslant(G, g) $. 这与 $ (M, m) $ 是极大元矛盾, 故 $ M = E $, 此时取 $ \tilde{f}=m $ 即可.\qed
	\end{Proof}
	\begin{Theorem}[Hahn-Banach]
		设 $ E $ 是 \K 上的线性空间,  $ F\subset E $ 是线性子空间, 并设 $ p:E\to\R $ 是半范数,  $ f:F\to\K $ 是线性泛函, 且在 $ F $ 上成立 $ \abs{f}\leqslant p $. 则存在线性泛函 $ \tilde{f}:E\to\R $ 使得
		\[
			(\tilde{f}|_{F}=f)\land(\forall x\in E\,(\abs{f(x)}\leqslant p(x))).
		\]
	\end{Theorem}
	\begin{Proof}
		当 $ \K = \R $ 时, 因为 $ F $ 上成立 $ f\leqslant p $, 由 Hahn-Banach 的实情形可知存在 $ \tilde{f}:E\to \R $ 使得 $ \tilde{f}|_{F}=f $, 且 $ \forall x\in E\,(\tilde{f}(x)\leqslant p(x)) $, 从而
		\[
			\tilde{f}(-x)\leqslant p(-x) = p(x).
		\]
		故 $ \forall x\in E\,(\abs{\tilde{f}(x)}\leqslant p(x)) $.

		当 $ \K=\C $ 时, 因为 $ f=\Re f+\imag \Im f $, 令 $ \varphi = \Re f $. 则 $ \varphi:E\to\R $ 是实线性泛函. 注意到
		\[
			f(\imag x)= \imag \Re f(x) - \Im f(x),
		\]
		从而 $ \varphi(\imag x)=-\Im f(x) $. 于是可以写成
		\[
			f(x)=\varphi(x)-\imag \varphi(\imag x).
		\]
		由 $ \abs{f}\leqslant p $ 知 $ \tilde{f}(x)=\tilde{\varphi}(x)-\imag\tilde{\varphi}(\imag x) $, 则 $ \tilde{f} $ 是复线性泛函, 则 $ \forall x\in F $, 有
		\[
			\tilde{f}(x)=\tilde{\varphi}(x)-\imag\tilde{\varphi}(\imag x)= \varphi(x)-\imag\varphi(\imag x)=f(x),
		\]
		即 $ \tilde{f}|_{F}=f $.

		而 $ \forall x\in E $, 令 $ \lambda=\sgn\tilde{f}(x) $, 有 $ \abs{\lambda}=1 $, 从而
		\[
			|\tilde{f}(x)|=\lambda\tilde{f}(x)=\tilde{f}(\lambda x)= \tilde{\varphi}(\lambda x)-\imag \tilde{\varphi}(\imag\lambda x)=\tilde{\varphi}(\lambda x)\leqslant p(\lambda x)=p(x).
		\]
		\qed
	\end{Proof}
	下面是 Hahn-Banach 定理的推论:
	\begin{Corollary}\label{cor:保范延拓}
		设 $ E $ 是 \K 上的赋范空间,  $ F $ 是 $ E $ 的线性子空间, $ f:F\to\K $ 是连续线性泛函, 则存在 $ f $ 的连续线性延拓 $ \tilde{f}:E\to\K $ 使得
		\[
			(\tilde{f}|_{F}=f)\land (\|\tilde{f}\|=\norm{f}),
		\]
		称 $ \tilde{f} $ 为 $ f $ 的\textbf{保范延拓}.
	\end{Corollary}
	\begin{Proof}
		因为 $ \forall x\in F $ 成立 $ \abs{f(x)}\leqslant\norm{f}\norm{x} $. 由 Hahn-Banach 定理可知存在 $ \tilde{f} $ 使得 $ \tilde{f}|_{F}=f $, 且
		\[
			\forall x\in E\,(|\tilde{f}(x)|\leqslant p(x)=\norm{f}\norm{x}),
		\]
		即 $ \|\tilde{f}\|\leqslant \norm{f} $, 由 $ \tilde{f}|_{F}=f $ 可知 $ \|\tilde{f}\|=\norm{f} $.\qed
	\end{Proof}
	\begin{Corollary}\label{cor:HB2}
		设 $ E $ 是赋范空间, $ x_{0}\in E $ 且 $ x_{0}\ne 0 $, 则 $ \exists f\in \Star{E} $ 使得 $ f(x_{0})=\norm{x_{0}} $ 且 $ \norm{f}=1 $.
	\end{Corollary}
	\begin{Proof}
		令 $ F=\K x_{0} $, 则定义 $ F $ 上的线性泛函 $ f_{0} $ 满足: $ f_{0}(\lambda x_{0})=\lambda\norm{x_{0}} $. 由推论~\ref{cor:保范延拓}~可知存在 $ f\in\Star{E} $ 使得 $ f|_{F}=f_{0} $ 且 $ \norm{f}=\norm{f_{0}}=1 $, 即 $ f(x_{0})=\norm{x_{0}} $.\qed
	\end{Proof}
	\begin{Corollary}\label{cor:HB3}
		设 $ E $ 是赋范空间, 则任取 $ x\in E $, 有
		\[
			\norm{x}=\sup\{ \abs{f(x)}:f\in\Star{E}, \norm{f}\leqslant1 \},
		\]
		且上确界可达.
	\end{Corollary}
	\begin{Proof}
		记等式右侧为 $ \alpha $, 则 $ \forall f\in \Star{E} $, 由 $ \abs{f(x)}\leqslant\norm{f}\norm{x} $ 可知 $ \alpha \leqslant\norm{x} $. 而由推论~\ref{cor:HB2}~可知对任意 $ x\in E, x\ne 0 $, 存在 $ f_{0}\in \Star{E} $, 使得 $ f_{0}(x)=\norm{x} $ 且 $ \norm{f_{0}}=1 $, 故 $ \alpha =\norm{x} $.\qed
	\end{Proof}
	推论~\ref{cor:HB3}~说明 $ x\in E $ 的范数可被 $ \Star{E} $ 中的元素表达, 考虑双线性泛函
	\[
		B:E\times\Star{E}\to\K\quad(x, f)\mapsto f(x).
	\]
	则 $ \forall(x, f)\in E\times\Star{E} $, 都有 $ \abs{B(x, f)}\leqslant\norm{x}\norm{f} $, 由推论~\ref{cor:双线性映射连续性}~知 $ B $ 连续. 从而对任意给定的 $ x\in E $, 都有 $ B(x, \cdot):\Star{E}\to\K $ 连续, 也即
	\[
		B(x,\cdot)\in\Star{(\Star{E})}=: E^{**}\quad x\mapsto B(x, \cdot)
	\]
	是等距同构, 此时记 $ E\hookrightarrow E^{**} $, 称 $ E $ 可\textbf{等距嵌入} $ E^{**} $.
