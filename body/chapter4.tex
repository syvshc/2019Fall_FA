% !TeX root = main.tex

\chapter{Hilbert空间上紧算子的谱理论}

\section{有限秩算子与紧算子}

	为了与之后算子代数的理论使用的符号衔接, 接下来用 $ x, y $ 等来表示算子而使用 $ \xi,\eta $ 等来表示线性空间中的元素.

	\begin{Definition}[有限秩算子, 紧算子]\index{Y!有限秩算子}\index{J!紧算子}
	设$ E, F $均为Banach空间, $ x : E\to F $是线性算子(在本章中以后直接称为算子).
	\begin{enumerate}[(1)]
	\item 若$ x : E\to F $连续, 且$ \dim x(E)<\infty $, 则称$ T $是\textbf{有限秩算子}, 其全体记作$ \Fr(E,F) $.
	\item 若$ x(B_E) $相对紧, 则称$ x $是\textbf{紧算子}, 其全体记作$ \CK(E,F) $.
	\end{enumerate}
	特别地, 若$ E=F $, 记$ \Fr(E)=\Fr(E,F) $, $ \CK(E)=\CK(E,F) $.
	\end{Definition}
	
	\begin{Example}~
	\begin{enumerate}[(1)]
	\item 若$ E $有限维, 则$ x : E\to F $是有限秩算子.
	\item 设$ H $是Hilbert空间, $ F\subset H $是有限维子空间, 且$ \{ \seq{\delta} \} $是$ F $的规范正交基, 令
	\[
	x : H\to F,\qquad \xi\mapsto\sum_{k=1}^n\lrangle{\xi,\delta_k}\delta_k
	\]
	则$ x $是有限秩的.
	\item 有限秩算子一定是紧算子, 这因$ \baro{x(B_E)}\subset\norm{x}\bar{B}_{x(E)} $是紧的.
	\item 设$ H $是Hilbert空间, $ (\epsilon_n)_{n\geqslant 1} $是$ H $的规范正交基, 令
	\[
	x : H\to H,\qquad \xi\mapsto\sum_{n\geqslant 1}\frac{1}{n}\lrangle{\xi,\epsilon_n}\epsilon_n
	\]
	则$ x $是紧的.
	\item 由(3)知$ \Fr(E,F)\subset\CK(E,F) $. 又因为相对紧集是有界集, 可知$ \CK(E,F)\subset\CB(E,F) $, 则
	\[
	\Fr(E,F)\subset\CK(E,F)\subset\CB(E,F).
	\]
	\end{enumerate}
	\end{Example}
	
	\begin{Definition}[理想]\index{L!理想}
	设$ \CB $是一个代数, $ \CA\subset\CB $, 若
	\[
	\forall x\in\CB\,\forall a\in\CA\,(ax\in\CA\land xa\in\CA)
	\]
	则称$ \CA $是$ \CB $的一个\textbf{理想}.
	\end{Definition}

	\begin{Theorem}
		$ \Fr(E, F) $ 与 $ \CK(E, F) $ 都是 $ \CB(E, F) $ 的线性子空间, 且 $ \Fr(E) $ 与 $ \CK(E) $ 都是 $ \CB(E) $ 的理想.
	\end{Theorem}
	\begin{Proof}
		(1) 任取 $ x_{1}, x_{2}\in\Fr(E, F) $, 则 $ \dim x_{1}(E)<\infty, \dim x_{2}(E)<\infty $, 由
		\[
			(x_{1} + x_{2})(E)\subset x_{1}(E)+x_{2}(E)
		\]
		可知 $ \dim(x_{1}+x_{2})(E)<\infty $, 故 $ x_{1}+x_{2}\in\Fr(E, F) $, 再任取 $ \lambda\in\K $, 由
		\[
			(\lambda x_{1})(E)=x_{1}(E)
		\]
		知 $ \dim(\lambda x_{1})(E)<\infty $, 故 $ \lambda x_{1}\in\Fr(E, F) $, 于是 $ \Fr(E, f) $ 是 $ \CB(E, F) $ 的线性子空间.
		
		再取 $ x_{1}, x_{2}\in \CK(E, F) $, 并定义
		\[
			\varphi: F\times F\to F\qquad (\xi, \eta)\mapsto \xi+\eta.
		\]
		则由 $ \norm{\varphi(\xi, \eta)}=\norm{\xi+\eta}\leqslant 2\max\{ \norm{\xi}, \norm{\eta} \} $ 知 $ \varphi $ 连续. 从而
		\[
			\varphi(\baro{x_{1}(B_{E})}\times \baro{x_{2}(B_{E}}))=\baro{x_{1}(B_{E})}+\baro{x_{2}(B_{E})}
		\]
		是紧集, 故 $ x_{1}(B_{E})+x_{2}(B_{E})\subset\baro{x_{1}(B_{E})}+\baro{x_{2}(B_{E})} $ 相对紧, 也即 $ x_{1}+x_{2}\in\CK(E, F) $. 又由 $ (\lambda x_{1})(B_{E})=\lambda x_{1}(B_{E}) $ 相对紧, 故 $ \lambda x_{1}\in\CK(E, F) $, 于是 $ \CK(E, F) $ 也是 $ \CB(E, F) $ 的线性子空间.

		(2) 任取 $ x\in\Fr(E), y\in\CB(E) $, 由
		\[
			(xy)(E) = x(y(E))\subset x(E)
		\]
		知 $ \dim(xy)(E)<\infty $, 即 $ xy\in\Fr(E) $. 再由
		\[
			(yx)(E)=y(x(E))
		\]
		与 $ y $ 的有界性知 $ \dim(yx)(E)<\infty $, 即 $ yx\in\Fr(E) $. 于是 $ \Fr(E) $ 是 $ \CB(E) $ 的理想.

		再任取 $ x\in\CK(E), y\in\CB(E) $. 则由
		\[
			(xy)(B_{E}) = x(y(B_{E}))\subset x(\norm{y}B_{E}) = \norm{y}x(B_{E})
		\]
		知 $ xy(B_{E}) $ 相对紧, 故 $ xy\in\CK(E) $. 再由
		\[
			yx(B_{E})=y(x(B_{E}))
		\] 
		与 $ y $ 的连续性知 $ (yx)(B_{E}) $ 相对紧. 故 $ yx\in\CK(E) $, 于是 $ \CK(E) $ 也是 $ \CB(E) $ 的理想.\qed
	\end{Proof}

	\begin{Theorem}
		设 $ x\in\CB(E, F) $, 则 $ x $ 紧的充分必要条件是 $ \Star{x} $ 紧.
	\end{Theorem}
	\begin{Proof}
		\textsl{必要性}. 设 $ x $ 是紧算子, 则 $ x(B_{E}) $ 相对紧. 需证 $ \Star{x}(B_{\Star{F}}) $ 相对紧. 只需证明对任意 $ (\epsilon_{n})_{n\geqslant1}\subset \Star{x}(B_{\Star{F}}) $ 都有收敛子列. 令 $ (\delta_{n})_{n\geqslant1}\subset B_{\Star{F}} $ 使得 $ \Star{x}(\delta_{n})=\epsilon_{n} $, 即 $ \delta_{n}\circ x=\epsilon_{n} $. 则由 $ \Star{x} $ 的连续性, 只需说明 $ (\delta_{n})_{n\geqslant1} $ 有收敛子列即可. 注意到 $ \delta_{n} $ 的定义域为 $ \baro{x(B_{E})} $, 这是一个紧集.

		使用 Arzel\`a-Ascoli 引理, 要说明 $ (\delta_{n})_{n\geqslant1} $ 等度连续, 且对任意 $ \xi\in\baro{x(B_{E})} $, 都有 $ (\delta_{n}(\xi))_{n\geqslant1} $ 相对紧. 由 $ (\delta_{n})_{n\geqslant1}\subset B_{\Star{F}} $ 知 $ (\delta_{n})_{n\geqslant1} $ 有界, 从而 $ (\delta_{n}(\xi))_{n\geqslant1} $ 有界. 于是 $ (\delta_{n}(\xi))_{n\geqslant1} $ 相对紧, 则 $ \forall\varepsilon>0, \forall \xi, \eta\in\baro{x(B_{E})} $, 若 $ \norm{\xi-\eta}<\varepsilon $, 则
		\[
			\abs{\delta_{n}(\xi)-\delta_{n}(\eta)} = \abs{\delta_{n}(\xi-\eta)}\leqslant\norm{\xi-\eta}<\varepsilon.
		\] 
		故 $ (\delta_{n})_{n\geqslant1} $ 等度连续, 从而 $ (\delta_{n})_{n\geqslant1} $ 相对紧, 则存在 Cauchy 子列,
		记作 $ (\delta_{n_{k}})_{k\geqslant1} $, 由
		\[
			\begin{aligned}
				\norm{\epsilon_{n_{k}}-\epsilon_{n_{j}}} & = \sup_{\eta\in B_{E}}\norm{\lrangle{\epsilon_{n_{k}}-\epsilon_{n_{j}}, \eta}} \\
				& = \sup_{\eta\in B_{E}}\norm{\lrangle{\Star{x}(\delta_{n_{k}}), \eta}-\lrangle{\Star{x}(\delta_{n_{j}}), \eta}}\\
				& = \sup_{\eta\in B_{E}}\norm{\lrangle{\delta_{n_{k}}, x\eta}-\lrangle{\delta_{n_{j}}, x\eta}}\\
				& =  \sup_{\xi\in x(B_{E})}\norm{\lrangle{\delta_{n_{k}}, \xi}-\lrangle{\delta_{n_{j}}, \xi}}\\
				& \leqslant \norm{\delta_{n_{k}}-\delta_{n_{j}}}_{\baro{x(B_{E})}}\to 0.
			\end{aligned}
		\]
		故 $ (\epsilon_{n_{k}})_{k\geqslant1} $ 也是 Cauchy 列. 由 $ \Star{E} $ 的完备性可知 $ (\epsilon_{n_{k}})_{k\geqslant1} $ 收敛, 故 $ \Star{x}(B_{\Star{F}}) $ 相对紧, 即 $ \Star{x} $ 是紧算子.

		\textsl{充分性}. 若 $ \Star{x} $ 是紧算子, 由必要性可知 $ x^{**} $ 是紧算子, 由 $ x = x^{**}|_{E} $ 可知 $ x(B_{E})\subset x^{**}(B_{E}) $ 相对紧, 故 $ x $ 是紧算子.\qed
	\end{Proof}

	\begin{Proposition}
		设 $ E $ 是 Banach 空间, 则 $ \baro{\Fr(E)}=\CB(E) $ 当且仅当 $ \dim E<\infty $.
	\end{Proposition}
	\begin{Proof}
		\textsl{充分性}. 因为 $ \dim E<\infty $, 故 $ \CB(E)=\Fr(E) $.

		\textsl{必要性}. 由 $ \baro{\Fr(E)}=\CB(E) $, 考虑恒同算子 $ \id_{E} $, 则存在 $ x\in\Fr(E) $ 使得 $ \norm{\id_{E}-x}<1 $, 由习题 3.9 结论知 $ \id_{E}-(\id_{E}-x)=x $ 可逆, 从而
		\[
			E = x^{-1}(x(E))
		\]
		是有限维的.\qed
	\end{Proof}

	\begin{Definition}[逼近性质]\index{B!逼近性质}\label{def:逼近性质}
		设 $ E $ 是 Banach 空间, 若 $ \baro{\Fr(E)}=\CK(E) $, 则称 $ E $ 具有\textbf{逼近性质}.
	\end{Definition}

	\begin{Proposition}
        Hilbert 空间具有逼近性质, 即 $ \baro{\Fr(H)}=\CK(H) $.
    \end{Proposition}
    \begin{Proof}
        设 $ \{e_i\}_{i\in I} $ 是 $ H $ 的一个正交基, 对 $ J\in\fin I $, 记 $ p_J $ 为对应的正交投影算子, 即
        \[
            p_J : H\to\Span\{e_j\}_{j\in J}
        \]
        从而 $ \forall J\in\fin I $, 都有 $ \norm{p_J}=1 $.

        若 $ x\in\CK(H) $, 则对任意的 $ J\in\fin I $ 都有 $ p_Jx\in\Fr(H) $, 于是只需要证明 $ \lim_{J\uparrow I}\norm{p_Jx-x}=0 $ 即可. 取 $ \varepsilon>0 $, 由于 $ x $ 紧, 可知存在 $ \seq[k]{\xi}\in B_H $ 使得
        \[
            \forall\xi\in B_H\,\exists n\in\{1,\dots,k\}\,(\norm{x(\xi)-x(\xi_n)}<\varepsilon),
        \]
        且 $ \forall\eta\in H $ 都有 $ \lim_{J\uparrow I}\norm{p_J\eta-\eta}=0 $, 于是
        \[
            \exists J_0\in\fin I\,(J\supset J_0\Longrightarrow\forall i\in\{1,\dots,k\}\,(\norm{p_Jx(\xi_i)-x(\xi_i)}<\varepsilon)).
        \]
        于是当 $ J\supset J_0 $ 时, 有
        \[
            \begin{aligned}
                \norm{p_Jx(\xi)-x(\xi)}&\leqslant\norm{p_Jx(\xi)-p_Jx(\xi_i)}+\norm{p_Jx(\xi_i)-x(\xi_i)}+\norm{x(\xi_i)-x(\xi)}\\
                &<\varepsilon+\varepsilon+\varepsilon\\
                &=3\varepsilon,
            \end{aligned}
        \]
        从而 $ x\in\baro{\Fr(H)} $.\qed
	\end{Proof}
	
	\begin{Proposition}
		对 Hilbert 空间 $ H $, $ \CK(H) $ 是 $ \CB(H) $ 的闭线性子空间.
	\end{Proposition}
	\begin{Proof}
		取 $ \CK(H) $ 中的一列 $ (x_n)_{n\geqslant 1} $ 收敛到 $ x\in\CB(H) $, 那么
        \[
            \forall\varepsilon>0\,\exists n_0\in\Zi\,\left(n\geqslant n_0\Longrightarrow \norm{x_n-x}<\frac{\varepsilon}{3}\right),
        \]
        而 $ x_n $ 紧, 从而 $ x_n(B_H) $ 是预紧的, 从而存在 $ F\in\fin B_H $ 使得
        \[
            \forall h\in B_H,\exists\eta\in F\,\left( \norm{x_n(h)-x_n(\eta)}<\frac{\varepsilon}{3} \right).
        \]
        于是此时
        \[
            \begin{aligned}
                \norm{x(h)-x(\eta)}&\leqslant\norm{x(h)-x_n(h)}+\norm{x_n(h)-x_n(\eta)}+\norm{x_n(\eta)-x(\eta)}\\
                &<\frac{\varepsilon}{3}+\frac{\varepsilon}{3}+\frac{\varepsilon}{3}\\
                &=\varepsilon.
            \end{aligned}
        \]
        即 $ x(B_H) $ 也是预紧的. 而 Banach 空间中预紧与相对紧等价, 从而 $ x(B_H) $ 相对紧, 即 $ x\in\CK(H) $.\qed
	\end{Proof}

\section{紧算子的谱性质}
	\begin{Definition}[谱]\label{def:谱}
		设 $ x\in\CB(E) $ 
		\begin{enumerate}[(1)]
			\item 令集合 
			\[
				\sigma(x) = \{ \lambda\in\K: \lambda\id-x\ \text{不可逆} \}
			\]
			称 $ \sigma(x) $ 为 $ x $ 的\textbf{谱集}\index{P!谱集}, 并称 $ \rho(x)=\K\sm\sigma(x) $ 为 $ x $ 的\textbf{预解集}\index{Y!预解集}.
			\item 记 $ \lambda\id-x $ 为 $ \lambda -x $. 若 $ \lambda -x $ 不是单射, 则 $ \exists \xi\in E, \xi\ne0 $ 使得
			\[
				(\lambda-x)(\xi)=0.
			\]
			也即 $ \lambda \xi=x \xi $, 则称 $ \lambda $ 为 $ x $ 的\textbf{特征值}\index{T!特征值}, 称 $ \ker(\lambda -x) $ 为 $ x $ 关于 $ \lambda $ 的\textbf{特征子空间}\index{T!特征子空间}, 并称非零向量 $ \xi\in\ker(\lambda-x) $ 为 $ x $ 相应于 $ \lambda $ 的\textbf{特征向量}\index{T!特征向量}.
			\item 对任意 $ \lambda\in\rho(x) $, 称 $ R(\lambda, x)=(\lambda-x)^{-1} $ 为 $ x $ 的\textbf{预解式}\index{Y!预解式}.
		\end{enumerate}
	\end{Definition}

	\begin{Example}
		几个谱集的例子
		\begin{enumerate}[(1)]
			\item 设 $ x = \left[\begin{smallmatrix}
				1 & 0 \\ 0 & 2
			\end{smallmatrix}\right] $, 则 $ \sigma(x)=\{ 1, 2 \} $;
			\item 对 $ f\in C[0, 1] $, 定义 $ M_{f}g =fg $, 其中 $ g\in C[0, 1] $, 则 $ M_{f}\in\CB(C[0, 1]) $, 且 $ \sigma(M_{f})=\{ f(t):t\in [0, 1] \} $;
			\item 设 $ x $ 满足 $ x\xi=\sum\limits_{n\geqslant1}\frac{1}{n}\lrangle{\xi, e_{n}}e_{n} $, 则 $ \sigma(x)=\{ 0, 1, \frac{1}{2}, \frac{1}{3}, \dots \} $.
		\end{enumerate}
	\end{Example}

	\begin{Proposition}
		$ \forall \lambda, \mu\in\rho(x) $, 有
		\[
			R(\lambda, x)-R(\mu, x)=(\mu-\lambda)R(\lambda, x)R(\mu, x) = (\mu-\lambda)R(\mu, x)R(\lambda, x),
		\]
		这被称为\textbf{预解方程}\index{Y!预解方程}.
	\end{Proposition}
	\begin{Proof}
		由
		\[
			\begin{aligned}
				(\mu-\lambda)R(\mu, x)R(\lambda, x) & = R(\lambda, x)(\mu-\lambda)R(\mu, x)\\
				& = (\lambda-x)^{-1}((\mu-x)-(\lambda-x))(\mu-x)^{-1}\\
				& = (\lambda-x)^{-1}-(\mu-x)^{-1}\\
				& = R(\lambda, x)-R(\mu, x)
			\end{aligned}
		\]
		即证.\qed
	\end{Proof}

	\begin{Theorem}[谱半径]\index{P!谱半径}\label{thm:谱半径}
		设 $ x\in\CB(E) $
		\begin{enumerate}[(1)]
			\item 极限 $ \lim\limits_{n\to\infty} \norm{x^{n}}^{1/n} $ 存在, 且
			\[
				\lim_{n\to\infty}\norm{x^{n}}^{1/n}=\inf_{n\geqslant1}\norm{x^{n}}^{1/n},
			\]
			并将其记作 $ r(x) $, 称为算子 $ x $ 的\textbf{谱半径}. 
			\item $ \sigma(x) $ 是 \K 中的紧集, 且 $ \sigma(x)\subset\{ \lambda\in\K:\abs{\lambda}\leqslant r(x) \} $. 
		\end{enumerate}
	\end{Theorem}
	\begin{Proof}
		(1) 易知$ \liminf\limits_{n\to\infty}\norm{x^n}^{1/n}\geqslant\inf\limits_{n\geqslant 1}\norm{x^n}^{1/n} $, 记右侧为$ a $, 往证$ \forall\varepsilon>0 $, 都有$ \limsup\limits_{n\to\infty}\norm{x^n}^{1/n}\leqslant a+\varepsilon $. 由下确界的定义, 有
		\[
		\forall\varepsilon>0\,\exists n_0\in\N\,(\norm{x^{n_0}}^{1/n_0}\leqslant a+\varepsilon)
		\]
		且由带余除法定理可知$ \forall n\in\N $, $ n\geqslant n_0 $, 都有
		\[
		n=q(n)\cdot n_0+r(n),
		\]
		其中$ q(n)\in\N,\ r(n)\in\N $且$ 0\leqslant r(n)<n_0 $. 于是
		\[
		\norm{x^n}^{1/n}=\norm{x^{q(n)\cdot n_0+r(n)}}^{1/n}=\left(\norm{x^{n_0}}^{1/n_0}\right)^{q(n)\cdot n_0/n}\cdot\norm{x}^{r(n)/n}
		\]
		在上式中令$ n\to\infty $, 注意到$ q(n)\cdot n_0/n\to 1 $而$ r(n)/n\to 0 $, 有
		\[
		\limsup_{n\to\infty}\norm{x^n}^{1/n}\leqslant\limsup_{n\to\infty}\left(\norm{x^{n_0}}^{1/n_0}\right)^{q(n)\cdot n_0/n}\cdot\norm{x}^{r(n)/n}\leqslant a+\varepsilon,
		\]
		令$ \varepsilon\to 0^+ $即可.
		
		(2) 反证法. 若存在$ \lambda\in\sigma(x) $使得$ \abs{\lambda}>r(x) $, 由$ r(x)=\lim\limits_{n\to\infty}\norm{x^n}^{1/n} $可知
		\[
		\exists 0<c<1\,\exists n_0\in\N\,(n\geqslant n_0\Rightarrow\norm{x^n}^{1/n}\leqslant c\abs{\lambda})
		\]
		从而$ \norm{\left(\frac{x}{\lambda}\right)^n}\leqslant c^n $, 这说明$ \sum\limits_{n\geqslant 0}\left(\frac{x}{\lambda}\right)^n $在$ \CB(E) $中收敛, 且$ (\lambda-x)^{-1}=\frac{1}{\lambda}\sum\limits_{n\geqslant 0}\left(\frac{x}{\lambda}\right)^n $, 矛盾. 从而$ \lambda\notin\sigma(x) $. 作映射
		\[
		f : \K\to\CB(E),\qquad \lambda\mapsto\lambda-x,
		\]
		易证$ f $连续, 再由习题3.9的结论可知$ GL(E) $是开集, 从而$ \rho(x)=f^{-1}(GL(E)) $是开集, 从而$ \sigma(x)=\K\sm\rho(x) $是闭集. 由上一结论可知$ \sigma(x) $是有界集, 从而$ \sigma(x) $是紧的.\qed
	\end{Proof}

	\begin{Theorem}[谱半径定理]\index{P!谱半径定理}\label{thm:谱半径定理}
		设 $ \K=\C, x\in\CB(E) $, 则 $ \sigma(x) $ 非空, 且
		\[
			r(x)=\sup_{\lambda\in\sigma(x)}\abs{\lambda}.
		\]
	\end{Theorem}
	\begin{Proof}
		\textsl{这是一个不正式的证明.}
		
		任取$ x\in\CB(E) $, 作映射
		\[
		R : \rho(x)\to\CB(E),\qquad \lambda\mapsto(\lambda-x)^{-1}
		\]
		由习题3.9的结论可知$ R $是连续的, 再取$ \xi\in\Star{\CB(E)} $, 定义
		\[
		\varphi : \rho(x)\to\C,\qquad \lambda\mapsto\xi(R(\lambda))
		\]
		那么$ \varphi $在$ \rho(x) $上连续. 任取$ \lambda_0\in\rho(x) $, 当$ \abs{\lambda-\lambda_0}<1/\norm{R(\lambda_0)} $时, 由
		\[
		\begin{aligned}
		(\lambda-x)^{-1}&=(\lambda-\lambda_0+\lambda_0-x)^{-1}\\
		&=R(\lambda_0)\sum_{n\geqslant 0}(-1)^nR(\lambda_0)^n(\lambda-\lambda_0)^n\\
		&=\sum_{n\geqslant 0}(-1)^nR(\lambda_0)^{n+1}(\lambda-\lambda_0)^n
		\end{aligned}
		\]
		可知$ \norm{(\lambda-\lambda_0)(\lambda_0-x)^{-1}}\leqslant 1 $, 从而上面的级数在$ \CB(E) $中绝对收敛, 继而收敛, 于是$ \varphi $在$ \rho(x) $上是全纯的.
		
		而当$ \abs{\lambda}\geqslant\norm{x} $时, 有$ R(\lambda)=(\lambda-x)^{-1}=\sum\limits_{n\geqslant 0}\frac{x^n}{\lambda^{n+1}} $, 它在$ \CB(E) $中绝对收敛, 从而$ \varphi(\lambda)=\sum\limits_{n\geqslant 0}\frac{\xi(x^n)}{\lambda^{n+1}} $, 于是
		\[
		\abs{\varphi(\lambda)}\leqslant\sum_{n\geqslant 0}\frac{\abs{\xi(x^n)}}{\lambda^{n+1}}\leqslant\sum_{n\geqslant 0}\frac{\norm{\xi}\norm{x}^n}{\lambda^{n+1}}=\frac{\norm{\xi}}{\lambda}\cdot\frac{1}{1-\norm{x}/\abs{\lambda}},
		\]
		令$ \lambda\to\infty $, 就有$ \varphi(\lambda)\to 0 $. 若$ \sigma(x)=\varnothing $, 那么$ \rho(x)=\C $, 此时由Liouville定理可知$ \varphi=0 $. 这意味着$ (\lambda-x)^{-1} $总是存在的, 从而对任意$ \xi\in\Star{\CB(E)} $, 都有$ \xi((\lambda-x)^{-1})=0 $. 但由Hahn-Banach定理可知存在$ \xi\in\Star{\CB(E)} $使得$ \xi((\lambda-x)^{-1})=0 $, 矛盾. 于是$ \sigma(x)\ne\varnothing $.
		
		由定理~\ref{thm:谱半径}~的(2)可知$ r(x)\geqslant\sup\limits_{\lambda\in\sigma(x)}\abs{\lambda} $, 将右侧记作$ \alpha $, 往证$ r(x)\leqslant\alpha $. 任取$ \lambda $使得$ \abs{\lambda}>\alpha $, 则$ \lambda\in\rho(x) $, 由$ \varphi(\lambda) $在$ \C\sm\bar{B}(0,\alpha) $全纯可知$ \varphi(\lambda) $绝对收敛. 记$ S_n=\frac{x^n}{\lambda^{n+1}} $后由一致有界原理可知$ (\norm{S_n})_{n\geqslant 1} $有界, 记$ M=\sup\limits_{n\geqslant 1}\norm{S_n} $, 那么
		\[
		\norm{x^n}^{1/n}\leqslant M^{1/n}\abs{\lambda}^{1+1/n},
		\]
		在上式中令$ n\to\infty $即可.\qed
	\end{Proof}

	\begin{Example}
		下面是一些算子和其谱半径的例子.
		\begin{enumerate}[(1)]
			\item 考虑对角矩阵 $ x=\diag\{ \seq{\lambda} \} $, 则 $ r(x)=\max\limits_{1\leqslant k\leqslant n}\abs{\lambda_{k}} $. 或使用定义, $ \forall m\geqslant1 $ 成立
			\[
				\norm{x^{m}}^{1/m}=\norm{\diag\{ \seq{\lambda^{m}} \}}^{1/m}=\max_{1\leqslant k\leqslant n}\abs{\lambda_{k}^{m}}^{1/m}=\max_{1\leqslant k\leqslant n}\abs{\lambda_{k}}.
			\]
			\item 考虑 Jordan 块
			\[
				x = J(0, 3)=\begin{bmatrix}
					0 & 1 & 0\\
					0 & 0 & 1\\
					0 & 0 & 0
				\end{bmatrix}
			\]
			则 $ r(x)=0 $, 此因 $ x^{3}=0 $. 进一步, 幂零矩阵的谱半径都是 0.
			\item 考虑算子 $ M_{f} $, 则
			\[
				r(M_{f}) = \lim_{n\to\infty}\norm{M_{f}^{n}}^{1/n}=\lim_{n\to\infty}\norm{M_{f^{n}}}^{1/n}=\lim_{n\to\infty}\norm{f^{n}}^{1/n}=\norm{f}.
			\]
			\item 考虑算子 $ x: \xi\mapsto\sum\limits_{n\geqslant1}\frac{1}{n}\lrangle{\xi, e_{n}}e_{n} $, 则 $ r(x)=1 $.
			\item 设 $ (e_{n})_{n\geqslant1} $ 是可分 Hilbert 空间 $ H $ 上的规范正交基, 考虑右移算子
			\[
				s: H\to H\qquad e_{n}\mapsto e_{n+1}\qquad (\forall n\geqslant1).
			\]
			由
			\[
				\Bnorm{s^{m}\sum_{n\geqslant1}\lambda_{n}e_{n}}=\Bnorm{\sum_{n\geqslant1}\lambda_{n}e_{n+m}}=\Big( \sum_{n\geqslant1}\abs{\lambda_{n}}^{2} \Big)^{1/2}
			\]
			知 $ \norm{s^{m}}\leqslant1 $. 又由 $ \norm{s^{m}e_{n}}=\norm{e_{n+m}} $ 知 $ \norm{s^{m}}=1 $, 故 $ r(s)=1 $. 
		\end{enumerate}
	\end{Example}

	\begin{Corollary}
		设 $ x\in\CB(E) $, 则 $ r(x)\leqslant\norm{x} $.
	\end{Corollary}
	\begin{Proof}
		由定义
		\[
			r(x)=\lim_{n\to\infty}\norm{x^{n}}^{1/n}\leqslant\lim_{n\to\infty}(\norm{x}^{n})^{1/n}=\norm{x}
		\]
		即证.\qed
	\end{Proof}
	
	\begin{Proposition}\label{prop:lambda-T的性质}
        设 $ x\in\CK(E) $, $ \lambda\in\C\sm\{0\} $, 则有
        \begin{enumerate}[(1)]
            \item $ \forall n\in\Zi\,(\dim\ker(\lambda-x)^n<\infty) $;
            \item $ \forall n\in\Zi $, $ \im(\lambda-x)^n $ 是闭的;
            \item $ \exists n\in\Zi\,\forall k\in\Zi\,(\im(\lambda-x)^n=\im(\lambda-x)^{n+k}) $.
        \end{enumerate}
    \end{Proposition}
    \begin{Proof}
        (1) 为符号简便记 $ K_n=\ker(\lambda-x)^n $, 由二项式定理有
        \[
            \begin{aligned}
                (\lambda-x)^n&=\sum_{k=0}^n(-1)^k\binom{n}{k}\lambda^{n-k}x^k\\
                &=\lambda^n-\sum_{k=1}^n(-1)^k\binom{n}{k}\lambda^{n-k}x^k=:\lambda^n+y,
            \end{aligned}
        \]
        因 $ \CK(E) $ 是 $ \CB(E) $ 的理想, 故 $ y\in\CK(E) $. 在 $ K_n $ 上有
        \[
            -\frac{1}{\lambda^n}y\xi=-\frac{1}{\lambda^n}((\lambda-x)^n-\lambda^n)\xi=-\frac{1}{\lambda^n}(\lambda-x)^n\xi+\xi=\xi,
        \]
        即 $ -y/\lambda^n|_{K_n}=\id_{K_n} $, 从而 $ \id_{K_n} $ 是紧的, 由于空间有限维当且仅当闭单位球紧, 故 $ \dim K_n<\infty $.

        (2) 为符号简便记 $ H_n=\im(\lambda-x)^n $, 先考虑 $ n=1 $ 的情形, 此时 $ E=K_1\oplus K_1^\bot $, 作
        \[
            \tilde{x} : E/K_1\to H_1,\qquad \xi+K_1\mapsto(\lambda-x)\xi,
        \]
        即 $ \tilde{x}=(\lambda-x)|_{K_1^\bot} $. 只需说明 $ \tilde{x}^{-1} $ 连续即可. 用反证法, 否则存在一列 $ (\xi_n)_{n\geqslant 1}\subset K_1^\bot $ 使得
        \[
            \forall n\in\Zi\,(\norm{\xi_n}=1)\land\Big(\lim_{n\to\infty}(\lambda-x)\xi_n=0\Big),
        \]
        由 $ x $ 是紧算子可知存在 $ (\xi_{n_k})_{k\geqslant 1}\subset(\xi_n)_{n\geqslant 1} $ 使得 $ (x\xi_{n_k})_{k\geqslant 1} $ 收敛, 记其极限为 $ \xi\in E $. 那么在
        \[
            \lambda\xi_{n_k}=x\xi_{n_k}+(\lambda-x)\xi_{n_k}
        \]
        中令 $ k\to\infty $ 就有 $ \lambda\xi=x\xi $, 即 $ \lim_{k\to\infty}\xi_{n_k}=\xi/\lambda $, 从而 $ \xi\in K_1 $. 而 $ K_1^\bot $ 是闭的, 于是只能
        \[
            \xi\in K_1\cap K_1^\bot=\{0\}\Longrightarrow\xi=0,
        \]
        这与 $ \norm{\xi}=\lim_{k\to\infty}\norm{\xi_{n_k}}=1 $ 矛盾.

        下面使用归纳法, 设 $ \seq{H} $ 都是闭的, 由
        \[
            H_{n+1}=(\lambda-x)^{n+1}(H)=(\lambda-x)H_n
        \]
        知对任意 $ \xi\in E $ 都有 $ x(\lambda-x)^n\xi=(\lambda-x)^nx\xi $, 从而 $ x(H_n)\subset H_n $. 类似于 $ n=1 $ 的情形可证 $ H_{n+1} $ 也是闭的.

        (3) 不妨
        \[
            H_1\subset H_2\subset\dots\subset H_n\subset\cdots,
        \]
        若所有包含关系均严格, 则总存在 $ \xi_n\in H_n $ 使得 $ \norm{\xi_n}=1 $ 且 $ \xi_n\in H_{n+1}^\bot $, 则对任意的 $ n,k\in\Zi $ 都有
        \[
            x\xi_n-x_\xi{n+k}=\lambda\left(\xi_n-\xi_{n+k}-\frac{1}{\lambda}(\lambda-x)(\xi_n-\xi_{n+k})\right),
        \]
        等式右侧括号内第一项属于 $ H_{n+1}^\bot $ 中而后面的项都落在 $ H_{n+1} $ 中(这因 $ (\lambda-x)H_n=H_{n+1} $), 从而
        \[
            \begin{aligned}
                \norm{x\xi_n-x\xi_{n+k}}&=\abs{\lambda}\norm{\xi_n}+\abs{\lambda}\norm{\xi_{n+k}+\frac{1}{\lambda}(\lambda-x)(\xi_n-\xi_{n+k})}\\
                &\geqslant\abs{\lambda}\norm{\xi_n}\\
                &=\abs{\lambda}.
            \end{aligned}
        \]
        即 $ (x\xi_n)_{n\geqslant 1} $ 没有收敛子列, 这与 $ x\in\CK(E) $ 是矛盾的.\qed
    \end{Proof}
	
	\begin{Remark}
		上一定理说明了紧算子具有某种意义上的有限性, 且其限制到闭集上几乎就是可逆的(模掉零空间后可逆).
	\end{Remark}

	\begin{Theorem}[紧算子的谱性质]\label{thm:紧算子的谱性质}
		设 $ x\in\CK(E) $, 则
		\begin{enumerate}[(1)]
			\item $ 0\in\sigma(x) $ 且 $ \sigma(x)\sm\{0\}\subset\sigma_p(x) $, 其中 $ \sigma_{p}(T) $ 是 $ T $ 的所有特征值的集合, 称为 $ T $ 的\textbf{点谱集};
			\item $ \sigma(x)\sm\{0\} $ 或者有限, 或者有极限点0.
		\end{enumerate}
	\end{Theorem}
	\begin{Proof}
		(1) 由 Riesz 引理可知无限维空间上的紧算子不可能可逆, 于是必有 $ 0\in\sigma(x) $. 再任取 $ \lambda\in\sigma(x)\sm\{0\} $, 只需证明 $ 	\lambda $ 是 $ x $ 的特征值. 若 $ \lambda-x $ 是单射, 则
		\[
			\exists n\in\Zi\,((\lambda-x)^{n+1}(E)=(\lambda-x)^n(E)),
		\]
		从而
		\[
			\forall\xi\in E\,\exists\eta\in E\,((\lambda-x)^{+1}\eta=(\lambda-x)^n\xi).
		\]

		由 $ \lambda-x $ 是单射可知 $ \xi=(\lambda-x)\eta $, 而由 $ \xi $ 的任意性可知 $ \lambda-x $ 是满射, 从而由开映射定理可知 $ \lambda-x $ 是	一个同构, 即 $ \lambda-x $ 可逆. 这与 $ \lambda\in\sigma(x) $ 是矛盾的, 故 $ \lambda-x $ 不可能是单射, 也即 $ \ker(\lambda-x)\ne\{0\} $.

		(2) 只需要说明 $ \forall\delta>0 $, 都只存在有限多个不同的 $ \lambda\in\sigma_p(x)\sm\{0\} $ 使得 $ \abs{\lambda}\leqslant\delta $ 即可. 用反证法, 若存在这样的 $ \delta>0 $ 使得 $ \seq{\lambda}\in\sigma_p(x) $ 满足 $ \abs{\lambda_i}>\delta $, 设 $ \xi_i $ 是对应地特征向量而 	$ E_n=\Span\{\xi_i\}_{i=1}^n $. 下面说明存在序列 $ (\eta_n)_{n\geqslant 1} $ 使得 $ (x\eta_n)_{n\geqslant 1} $ 没有收敛子列. 注意到
		\[
			E_1\subset E_2\subset\dots\subset E_n\subset\cdots,
		\]
		且上述所有包含关系都是严格的. 从而存在 $ \eta_n\in E_n $ 使得 $ \norm{\eta_n}=1 $ 且 $ \eta_n\in E_{n+1}^\bot $, 于是对任意 $ m,n\in\Zi $ 满足 $ m>n $, 都有
		\[
			\begin{aligned}
				\norm{x\eta_m-x\eta_n}&=\norm{-\lambda_{m+1}\eta_m+x\eta_m-x\eta_n+\lambda_{m+1}\eta_m}\\
				&\geqslant\abs{\lambda_{m+1}}\norm{\eta_m}\\
				&=\abs{\lambda}\\
				&>\delta,
			\end{aligned}
		\]
		其中第一个不等号因右侧范数内最后一项落在 $ E_{m+1}^\bot $ 内而前面各项均落在 $ E_m $ 内. 从而 $ (x\eta_n)_{n\geqslant 1} $ 无收敛子列, 这与 $ x\in\CK(E) $ 矛盾.\qed
	\end{Proof}
	
\section{Hilbert空间上的正规紧算子}

	\subsection{自伴算子与正算子}
	
	\begin{Definition}[自伴算子, 正算子]\index{Z!自伴算子}\index{Z!正算子}
	若$ T\in\CB(H) $且$ \Star{T}=T $, 则称$ T $是\textbf{自伴算子}. 若$ T $是自伴的且
	\[
	\forall \xi\in H\,(\lrangle{T\xi,\xi}\geqslant 0)
	\]
	则称$ T $是\textbf{正算子}.
	\end{Definition}
	
	\begin{Example}
	考虑最简单的两种算子: 即$ n $阶复矩阵和乘法算子:
	\begin{enumerate}[(1)]
	\item 设$ A\in\mathbb{M}_n(\C) $, 则$ A $是自伴的当且仅当$ A $是Hermite的, 也即$ A^\dagger=A $; $ A $是正的当且仅当$ A $是半正定的.
	\item 设$ f\in C[0,1] $, $ M_f\in\CB(L_2(0,1)) $, 那么$ M_f $是自伴的当且仅当$ \bar{f}=f $; $ M_f $是正的当且晋档$ f\geqslant 0 $.
	\end{enumerate}
	\end{Example}
	
	\begin{Remark}
	(1) 若$ x $是自伴的, 则由
	\[
	\baro{\lrangle{x\xi,\xi}}=\lrangle{\xi,x\xi}=\lrangle{\Star{x}\xi,\xi}=\lrangle{x\xi,\xi}
	\]
	可知$ \lrangle{x\xi,\xi}\in\R $, 但反之未必成立.
	
	(2) 若$ x $是正算子, 则映射$ \lrangle{\xi,\eta}\mapsto\lrangle{x\xi,\eta} $具有除正定性以外的所有内积的性质, 故有Cauchy-Schwarz不等式成立:
	\[
	\abs{\lrangle{x\xi,\eta}}^2\leqslant\lrangle{x\xi,\xi}\lrangle{x\eta,\eta},\qquad\forall \xi,\eta\in H
	\]
	从而
	\[
	\norm{x}=\sup_{\norm{\xi}=1}\norm{x\xi}=\sup_{\norm{\xi}=\norm{\eta}=1}\abs{\lrangle{x\xi,\xi}}^{1/2}\abs{\lrangle{x\eta,\eta}}^{1/2}=\sup_{\norm{\xi}=1}\abs{\lrangle{x\xi,\xi}},
	\]
	故$ x $是正算子时, 有$ \norm{x}=\sup\limits_{\norm{\xi}=1}\abs{\lrangle{x\xi,\xi}} $.
	\end{Remark}
	
	\begin{Theorem}\label{thm:自伴算子的谱性质}
	设$ x $是$ H $上的自伴算子, 则:
	\begin{enumerate}[(1)]
	\item $ r(x)=\norm{x}=\sup\limits_{\norm{\xi}=1}\abs{\lrangle{x\xi,\xi}} $;
	\item 令$ m=\inf\limits_{\norm{\xi}=1}\lrangle{x\xi,\xi} $, $ M=\sup\limits_{\norm{\xi}=1}\lrangle{x\xi,\xi} $, 那么$ \sigma(x)\subset[m,M] $, 且$ m\in\sigma(x) $, $ M\in\sigma(x) $, 并且有
	\[
	r(x)=\norm{x}=\max\{\abs{m},\abs{M}\}=\max_{\lambda\in\sigma_p(x)}\abs{\lambda}.
	\]
	\end{enumerate}
	\end{Theorem}
	\begin{Proof}
	(1) 先证$ r(x)=\norm{x} $. 由$ x $自伴可知$ \Star{x}=x $, 即$ \forall \xi\in H,\norm{\xi}=1 $都有
	\[
	\norm{x\xi}^2=\lrangle{x\xi,x\xi}=\lrangle{\xi,\Star{x}x\xi}\leqslant\norm{x^2}\norm{\xi}^2,
	\]
	故$ \norm{x^2}\leqslant\norm{x}^2\leqslant\norm{x^2} $, 也即$ \norm{x^2}=\norm{x}^2 $成立. 重复以上过程, 有$ \norm{x}^{2^n}=\tnorm{x^{2^n}} $, 故
	\[
	r(x)=\lim_{n\to\infty}\norm{x^n}^{1/n}=\lim_{n\to \infty}\tnorm{x^{2^n}}^{2^{-n}}=\norm{x}.
	\]
	
	下面说明$ \sup\limits_{\norm{\xi}=1}\abs{\lrangle{x\xi,\xi}}=\norm{x} $, 记左侧为$ \alpha $, 则$ \alpha\leqslant\norm{x} $是显然的. 往证反向不等式成立, 则$ \forall \xi\in H $, 有
	\[
	\abs{\lrangle{x\xi,\xi}}=\norm{\xi}^2\abs{\lrangle{x\sgn \xi,\sgn \xi}}\leqslant\alpha\norm{\xi}^2,
	\]
	在上式中将$ \xi $替换为$ \xi+\lambda \eta $和$ \xi-\lambda \eta $, 有
	\[
	\begin{aligned}
	\abs{\lrangle{x(\xi+\lambda \eta),\xi+\lambda \eta}}&\leqslant\alpha\norm{\xi+\lambda \eta}^2\\
	\abs{\lrangle{x(\xi-\lambda \eta),\xi-\lambda \eta}}&\leqslant\alpha\norm{\xi-\lambda \eta}^2
	\end{aligned}
	\]
	由平行四边形法则
	\[
	\abs{\lrangle{x(\xi+\lambda \eta),\xi+\lambda \eta}-\lrangle{x(\xi-\lambda \eta),\xi-\lambda \eta}}=4\abs{\Re\bar{\lambda}\lrangle{x\xi,\eta}}
	\]
	另一方面
	\[
	\begin{aligned}
	\abs{\lrangle{x(\xi+\lambda \eta),\xi+\lambda \eta}-\lrangle{x(\xi-\lambda \eta),\xi-\lambda \eta}}&\leqslant\alpha(\norm{\xi+\lambda \eta}^2+\norm{\xi-\lambda \eta}^2)\\
	&=2\alpha(\norm{\xi}^2+\abs{\lambda}^2\norm{\eta}^2)
	\end{aligned}
	\]
	故
	\[
	2\abs{\Re\bar{\lambda}\lrangle{x\xi,\eta}}\leqslant\alpha(\norm{\xi}^2+\abs{\lambda}^2\norm{\eta}^2).
	\]
	再由$ \lambda $的任意性, 取$ \lambda=\sgn\baro{\lrangle{x\xi,\eta}} $, 有
	\[
	\abs{\lrangle{x\xi,\eta}}\leqslant\frac{\alpha}{2}(\norm{\xi}^2+\norm{\eta}^2)
	\]
	再令$ \eta=\xi $, 就有$ \abs{\lrangle{x\xi,\xi}}\leqslant\alpha $, 故$ \abs{\lrangle{x\xi,\xi}}=\alpha $.
	
	(2) 设$ \lambda\in\K $, 且$ \lambda\notin[m,M] $, 记$ d(\lambda)=d(\lambda,[m,M])>0 $. 对$ \norm{\xi}=1 $, 有
	\[
	\abs{\lrangle{(\lambda-x)\xi,\xi}}=\abs{\lambda-\lrangle{x\xi,\xi}}\geqslant d(\lambda)>0,
	\]
	故
	\[
	d(\lambda)\norm{\xi}^2\leqslant\abs{\lrangle{(\lambda-x)\xi,\xi}}\leqslant\norm{(\lambda-x)\xi}\norm{\xi},
	\]
	即$ d(\lambda)\norm{\xi}\leqslant\norm{(\lambda-x)\xi} $, 令$ (\lambda-x)\xi=0 $知$ \xi=0 $, 再取$ ((\lambda-x)\xi_n)_{n\geqslant 1} $是Cauchy列, 可知$ (\xi_n)_{n\geqslant 1} $也是Cauchy列, 从而$ (\lambda-x)(H) $是闭的. 故由
	\[
	(\lambda-x)(H)=\baro{(\lambda-x)(H)}=\ker (\bar{\lambda}-\Star{x})^\bot
	\]
	与$ \bar{\lambda}-\Star{x}=\bar{\lambda}-x $可知$ \lambda\notin[m,M]\Rightarrow\bar{\lambda}\notin[m,M] $, 也可验证$ \bar{\lambda}-x $是单射, 故$ \ker(\bar{\lambda}-x)=\{0\} $, 于是$ (\lambda-x)(H) $在$ H $中稠密, 即$ (\lambda-x)(H)=H $, 故$ \lambda\in\rho(x) $. 这说明$ \sigma(x)\subset[m,M] $.
	
	再证明$ m\in\sigma(x) $, 由$ m $的定义可知存在$ (\xi_n)_{n\geqslant}\subset H $且$ \norm{\xi_n}=1 $使得$ m=\lim\limits_{n\to\infty}\lrangle{x\xi_n,\xi_n} $, 则有
	\[
	\lim_{n\to \infty}\lrangle{(x-m)\xi_n,\xi_n}=\lim_{n\to\infty}\lrangle{x\xi_n,\xi_n}-m=0,
	\]
	由$ m $的定义可知$ x-m $是正的, 由Cauchy-Schwarz不等式可知
	\[
	\abs{\lrangle{(x-m)\xi_n,\eta}}\leqslant\lrangle{(x-m)\xi_n,\xi_n}^{1/2}\lrangle{(x-m)\eta,\eta}^{1/2}\leqslant\lrangle{(x-m)\xi_n,\xi_n}^{1/2}\norm{x-m}^{1/2}\norm{\eta},
	\]
	令$ \eta=(x-m)\xi_n $, 则有
	\[
	\norm{(x-m)\xi_n}^2\leqslant\lrangle{(x-m)\xi_n,\xi_n}^{1/2}\norm{x-m}^{1/2}\norm{(x-m)\xi_n}
	\]
	即
	\[
	\norm{(x-m)\xi_n}\leqslant\lrangle{(x-m)\xi_n,\xi_n}^{1/2}\norm{x-m}^{1/2}\to 0,
	\]
	于是$ (x-m)\xi_n\to 0 $, 这说明$ x-m $不可逆, 于是$ m\in\sigma(x) $. 类似可证$ M-x $是正算子, 不可逆, 从而$ M\in\sigma(x) $. 由谱半径定理可知$ r(x)=\max\{ \abs{m},\abs{M} \} $.\qed
	\end{Proof}
	
	\begin{Corollary}
	设$ x $是自伴算子, 则$ x $是正算子当且仅当$ \sigma(x)\subset[0,\infty) $, 且若$ x $是正算子, 则$ \norm{x}\in\sigma(x) $.
	\end{Corollary}
	\begin{Proof}
	若$ x $是正算子, 由$ \forall \xi\in H\,(\lrangle{x\xi,\xi}\geqslant 0) $可知$ m\geqslant 0,\ M\geqslant 0 $, 从而
	\[
	\sigma(x)\subset[m,M]\subset[0,\infty).
	\]
	反之, 若$ \sigma(x)\subset[0,\infty) $, 那么$ m\geqslant 0,\ M\geqslant 0 $, 故
	\[
	\forall \xi\in H(\lrangle{x\xi,\xi}\geqslant 0)
	\]
	即$ x $是正的.
	
	若$ x $是正算子, 由$ M=r(x)=\norm{x} $可知$ M\in\sigma(x) $.\qed
	\end{Proof}
	
	\begin{Corollary}
	设$ x $是$ H $上的自伴紧算子, 则存在$ x $的特征值$ \lambda $使得$ \abs{\lambda}=\norm{x} $.
	\end{Corollary}
	\begin{Proof}
	由定理~\ref{thm:自伴算子的谱性质}~的(2)可知$ \norm{x}=\max\limits_{\lambda\in\sigma(x)}\abs{\lambda} $, 再由定理~\ref{thm:紧算子的谱性质}~可知$ \forall\lambda\in\sigma(x)\sm\{0\} $, 都有$ \lambda\in\sigma_p(x) $, 故命题得证.\qed
	\end{Proof}
	
	\subsection{自伴紧算子的谱分解}
	
	\begin{Definition}[Hilbert空间的直和]\index{Z!直和}
	设$ \{H_i\}_{i\in\alpha} $是一组Hilbert空间, 令$ \bigoplus\limits_{i\in\alpha}H_i $是$ \prod\limits_{i\in\alpha}H_i $的子集, 其中$ (\xi_i)_{i\in\alpha}\in\bigoplus\limits_{i\in\alpha}H_i $使得$ \sum\limits_{i\in\alpha}\norm{\xi_i}^2<\infty $, 即
	\[
	\bigoplus_{i\in\alpha}H_i=\left\{ (\xi_i)_{i\in\alpha} : \sum_{i\in\alpha}\norm{\xi_i}^2<\infty \right\}
	\]
	定义其上的加法和数乘
	\[
	(\xi_i)_{i\in\alpha}+(\eta_i)_{i\in\alpha}:=(\xi_i+\eta_i)_{i\in\alpha},\qquad\lambda(\xi_i)_{i\in\alpha}:=(\lambda \xi_i)_{i\in\alpha},
	\]
	并赋予范数
	\[
	\norm{(\xi_i)_{i\in\alpha}}=\Big( \sum_{i\in\alpha}\norm{\xi_i}^2 \Big)^{1/2},
	\]
	它可以被以下的内积诱导
	\[
	\lrangle{(\xi_i)_{i\in\alpha},(\eta_i)_{i\in\alpha}}=\sum_{i\in\alpha}\lrangle{\xi_i,\eta_i},
	\]
	则由各$ H_i $是Hilbert空间可知$ \bigoplus\limits_{i\in\alpha}H_i $也是Hilbert空间, 并称其为$ \{ H_i \}_{i\in\alpha} $的\textbf{直和}.
	\end{Definition}
	
	\begin{Theorem}[自伴紧算子的谱分解]\label{thm:自伴紧算子的谱分解}
	设$ x $是$ H $上的自伴紧算子, $ V_\lambda $表示对应于$ \lambda $的特征子空间, 则
	\begin{enumerate}[(1)]
	\item $ H=\bigoplus\limits_{\lambda\in\sigma_p(x)}V_\lambda $, 即$ H $有由$ x $的特征向量构成的正交基.
	\item 空间$ \baro{x(H)} $有一个由特征向量$ (e_n)_{n\geqslant 1} $构成的正交基, 此处$ (e_n)_{n\geqslant 1} $是分别对应于特征值$ (\lambda_n)_{n\geqslant 1} $的特征向量(这里序列$ (\lambda_n)_{n\geqslant 1} $也可以是有限的)使得
	\[
	\forall \xi\in H\,\Big(x\xi=\sum_{n\geqslant 1}\lambda_n\lrangle{\xi,e_n}e_n\Big)
	\]
	在范数意义下收敛, 且若$ (\lambda_n)_{n\geqslant 1} $无限, 则成立$ \lim\limits_{n\to\infty}\lambda_n=0 $.
	\end{enumerate}
	\end{Theorem}
	\begin{Proof}
	(1) 由$ x $是紧算子可知$ \sigma(x) $之多可数, 记作
	\[
	\sigma(x)=\{ \seq{\lambda},\dots \}
	\]
	且不妨$ \abs{\lambda_1}\geqslant\abs{\lambda_2}\geqslant\cdots\geqslant\abs{\lambda_n}\geqslant\cdots $, 其中序列中的特征值可以重复, 重复次数即为其代数重数. 设$ \lambda $和$ \mu $是$ x $的不同特征值, 则存在非零的$ \xi,\eta $使得$ x\xi=\lambda \xi $且$ x\eta=\mu \eta $. 不妨设$ \lambda\ne 0 $, 则
	\[
	\lambda\lrangle{\xi,\eta}=\lrangle{x\xi,\eta}=\lrangle{\xi,x\eta}=\mu\lrangle{\xi,\eta}
	\]
	可知$ (\lambda-\mu)\lrangle{\xi,\eta}=0 $, 由$ \lambda=\mu\ne0 $可知只能$ \lrangle{\xi,\eta}=0 $, 故$ \bigoplus\limits_{\lambda\in\sigma_p(x)}V_\lambda\subset H $.
	
	再令$ \tilde{V}=\Big(\bigoplus\limits_{\lambda\in\sigma_p(x)}V_\lambda\Big)^\bot $, 则需证明$ \forall \zeta\in\tilde{V} $都有$ x\zeta\in\tilde{V} $. 若$ \lambda\ne 0 $, 则有
	\[
	\forall \eta\in V_\lambda\,(\lambda\lrangle{\zeta,\eta}=0)\Rightarrow\lrangle{\zeta,x\eta}=0\Rightarrow\lrangle{x\zeta,\eta}=0\Rightarrow x_\zeta\bot V_\lambda.
	\]
	若$ \lambda=0 $, 则
	\[
	\forall \eta\in V_0\,(\lrangle{x\zeta,\eta}=\lrangle{\zeta,x\eta}=0)\Rightarrow x\zeta\bot V_0,
	\]
	从而$ x\zeta\in\tilde{V} $. 故$ x|_{\tilde V} $有意义且为紧算子, 从而$ \sigma(x|_{\tilde V}) $非空, 若有$ \lambda\in\sigma(x|_{\tilde V}) $非零, 则存在非零向量$ \xi\in\tilde{V} $使得$ x\xi=\lambda \xi $, 故$ \lambda\in\sigma_p(x) $, 这与$ \tilde{V} $的定义矛盾. 故只能$ \lambda=0 $, 即$ \sigma(x|_{\tilde{V}})=\{0\} $, 由谱半径定理可知$ \norm{x}=r(x)=0 $, 从而$ x|_{\tilde V}=0 $, 即$ \tilde{V}=\{0\} $, 也即
	\[
	\bigoplus_{\lambda\in\sigma_p(x)}V_\lambda=H.
	\]
	
	(2) 因$ \baro{x(H)}=\bigoplus\limits_{\lambda\in\sigma_p(x)\sm\{0\}}V_\lambda $, 于是有一个由特征向量$ (e_n)_{n\geqslant 1} $构成的正交基, 此时特征向量$ (e_n)_{n\geqslant 1} $对应于特征值$ (\lambda_n)_{n\geqslant 1} $, 使得
	\[
	\forall \xi\in H\,\bigg(x\xi=\sum_{n\geqslant 1}\lambda_n\lrangle{\xi,e_n}e_n\bigg)
	\]
	并且若$ (\lambda_n)_{n\geqslant 1} $无限, 有$ \lim\limits_{n\to\infty}\lambda_n=0 $.\qed
	\end{Proof}

	\subsection{正规紧算子的谱分解}

	这一小节主要目的是推广以下结论: 在 $ \K^n $ 上每个正规算子在某个正交基下都是一个对角阵. 对 Hilbert 空间 $ H $, 对应的结论是每个正规紧算子都可写成其谱点对应地特征子空间的直和(谱分解).

	下面做一些符号上的约定: 接下来总是假设 $ H $ 可分且 $ \dim H\geqslant \aleph_0 $; 对 $ x\in\CB(H) $, 用 $ \sigma(x) $ 记其谱集而用 $ \sigma_p(x) $ 记其点谱集. 对 $ \lambda\in\sigma_p(x) $, 以 $ V_\lambda(x)=\ker(\lambda-x) $ 记其特征子空间.
	
	\begin{Definition}[正规算子]\index{Z!正规算子}
        设 $ x\in\CB(H) $, 若有 $ x\Star{x}=\Star{x}x $ 成立, 则称 $ x $ 是一个\textbf{正规算子}.
	\end{Definition}
	
	\begin{Proposition}
        设 $ x\in\CB(H) $ 是一个正规算子, 则
        \begin{enumerate}[(1)]
            \item $ \ker x=\ker \Star{x} $;
            \item $ \lambda\in\sigma_p(x) $ 当且仅当 $ \bar{\lambda}\in\sigma_p(\Star{x}) $, 且 $ V_{\lambda}(x)=V_{\bar{\lambda}}(\Star{x}) $;
            \item $ (\lambda,\mu\in\sigma_p(x)\land\lambda\ne\mu)\Longrightarrow(V_\lambda(X)\perp V_\mu(x)) $.
        \end{enumerate}
    \end{Proposition}
    \begin{Proof}
        (1) 此因 $ \forall\xi\in H $ 时, 有
        \[
            \begin{aligned}
                \norm{x\xi}^2=\lrangle{x\xi,x\xi}&=\lrangle{\Star{x}x\xi,\xi}\\
                &=\lrangle{x\Star{x}\xi,\xi}=\lrangle{\Star{x}\xi,\Star{x}\xi}=\norm{\Star{x}\xi}^2
            \end{aligned}
        \]
        故 $ x\xi=0 $ 当且仅当 $ \Star{x}\xi=0 $.

        (2) 与(1)同理可证 $ \forall\xi\in H $, 有
        \[
            \norm{(\lambda-x)\xi}^2=\norm{\Star{(\lambda-x)}\xi}^2\norm{(\bar{\lambda}-\Star{x})\xi}^2
        \]

        (3) 设 $ \xi\in V_\lambda(x) $ 而 $ \eta\in V_\mu(x) $, 因 $ \lambda\ne\mu $, 由
        \[
            \lambda\lrangle{\xi,\eta}=\lrangle{\lambda\xi,\eta}=\lrangle{x\xi,\eta}=\lrangle{\xi,\Star{x}\eta}=\lrangle{\xi,\bar{\mu}\eta}=\mu\lrangle{\xi,\eta}
        \]
        可知 $ \lrangle{\xi,\eta}=0 $.\qed
	\end{Proof}
	
	以上命题是正规矩阵的性质推广到正规算子的情形, 而本节的主要结论, 即正规紧算子的谱分解如下;

    \begin{Theorem}[正规紧算子的谱分解]
        设 $ x\in\CK(H) $ 正规, 则
        \begin{enumerate}[(1)]
            \item $ \exists\lambda\in\sigma_p(x)\,(\norm{x}=\abs{\lambda}) $;
            \item $ H=\bigoplus_{\lambda\in\sigma_p(x)}V_\lambda(x) $, 这称为\textbf{谱分解};
            \item 存在 $ H $ 的正交基 $ (e_n)_{n\geqslant 0} $ 和复数列 $ (\lambda_n)_{n\geqslant 0} $, 其中 $ \lambda_n $ 是 $ x $ 的特征值, 使得对任意 $ \xi\in H $ 都有
            \[
                x\xi=\sum_{n\geqslant 0}\lambda_n\lrangle{e_n,\xi}e_n,\qquad \Star{x}\xi=\sum_{n\geqslant 0}\bar{\lambda}_n\lrangle{e_n,\xi}e_n
            \]
            成立, 且两级数均在 $ H $ 中收敛.
        \end{enumerate}
    \end{Theorem}
    \begin{Proof}
        (1) 记 $ x $ 的谱半径为 $ r(x):=\sup_{\lambda\in\sigma(x)}\abs{\lambda} $, 因 $ x $ 是正规算子, 故 $ \norm{x}=r(x) $ (相关结论见算子代数部分), 而 $ \sigma(x) $ 是紧集, 即
        \[
            \exists\lambda_0\in\sigma(x)\,(\norm{x}=\abs{\lambda_0}),
        \]
        当 $ \lambda_0=0 $ 时有 $ x=0 $ 且 $ 0\in\sigma_p(x) $, 若 $ \lambda_0\ne 0 $, 因 $ \sigma(x)\sm\{0\}\subset\sigma_p(x) $ 知 $ \lambda_0\in\sigma_p(x) $.
        
        (2) 为了符号简便, 不妨记右侧空间为 $ E $. 由 $ V_\lambda(x)=V_{\bar{\lambda}}(\Star{x}) $ 是 $ x,\Star{x} $ 的不变子空间可知 $ E $ 也是 $ x,\Star{x} $ 的不变子空间, 再由
        \[
            \forall\xi\in E^\bot\,\forall\eta\in E\,(\lrangle{x\xi,\eta}=\lrangle{\xi,\Star{x}\eta}=0)
        \]
        可知 $ E^\bot $ 也是 $ x,\Star{x} $ 的不变子空间, 于是只需要证明 $ E^\bot=\{0\} $ 即可.

        用反证法, 不妨设 $ E^\bot\ne\{0\} $ 且 $ \tilde{x}=x|_{E^\bot} $, 则 $ \tilde{x}\in\CK(H) $ 且 $ \Star{\tilde{x}}=\Star{x}|_{E^\bot} $, 从而由 $ x $ 的正规性可知 $ \tilde{x} $ 也是正规的, 于是在 $ E^\bot $ 上存在对应于特征值 $ \mu $ 的特征向量 $ e $, 矛盾.

        (3) 任取 $ V_\lambda(x) $ 的正交基 $ \{e_n^{(\lambda)}\}_{1\leqslant n\leqslant i_\lambda} $, 若 $ \lambda\ne 0 $ 则 $ i_\lambda $ 有限, 而 $ \{V_\lambda(x)\}_{\lambda\in\sigma_p(x)} $ 是两两正交的, 于是 $ \bigcup_{\lambda\in\sigma_p(x)}\{e_n^{(\lambda)}\}_{1\leqslant n\leqslant i_\lambda} $ 就是 $ H $ 的正交基, 它与相应的特征值(计重数)序列即满足要求.\qed
	\end{Proof}

	\section{部分等距算子\ \ 极分解}

	\begin{Definition}[部分等距算子]\index{B!部分等距算子}
        设 $ u\in\CB(H) $, 若存在 $ H $ 的闭子空间 $ E $ 使得
        \[
            \forall\xi\in E\,(\norm{u\xi}=\norm{\xi})\land\forall\xi\in E^\bot\,(u\xi=0),
        \]
        则称 $ u $ 是一个\textbf{部分等距算子}, 且称 $ E $ 是 $ u $ 的支撑.
    \end{Definition}

    \begin{Proposition}
        设 $ u\in\CB(H) $ 是支撑为 $ E $ 的部分等距, 则 $ \im u $ 是闭的, 且
        \begin{enumerate}[(1)]
            \item $ \Star{u} $ 是支撑为 $ \im u $ 的部分等距;
            \item $ \Star{u}u=p_E $ 且 $ u\Star{u}=p_{\im u} $;
            \item $ u|_E $ 是 $ E $ 到 $ \im u $ 的等距同构且 $ \Star{(u|_E)}=\Star{u}|_{\im u} $.
        \end{enumerate}
    \end{Proposition}
    \begin{Proof}
        因 $ u $ 在 $ E $ 上等距, 对任意 $ (x_n)_{n\geqslant 1}\subset u(E) $ 使得 $ x_n\to x $, 都存在 $ (y_n)_{n\geqslant 1}\subset E $ 使得 $ x_n=uy_n $. 因此
        \[
            \norm{x}=\lim_{n\to\infty}\norm{x_n}=\lim_{n\to\infty}\norm{uy_n}=\lim_{n\to\infty}\norm{y_n},
        \]
        即 $ (y_n)_{n\geqslant 1} $ 有界. 从而由序列紧性可知存在 $ (y_{n_k})_{k\geqslant 1}\subset(y_n)_{n\geqslant 1} $ 使得 $ y_{n_k}\to y $. 由 $ E $ 是闭的可知 $ y\in E $, 从而 $ uy=x\in u(E) $, 即 $ u(E) $ 是闭的.

        (1) 由 $ u|_E $ 是等距和 $ \im u $ 闭可知 $ \forall\xi,\eta\in E $ 都有
        \[
            \lrangle{\Star{u}u\xi,\eta}=\lrangle{u\xi,u\eta}=\lrangle{\xi,\eta}.
        \]
        置 $ \eta=\xi $ 有
        \[
            \norm{\Star{u}u\xi}=\norm{\xi}=\norm{u\xi}.
        \]
        从而 $ \Star{u}|_{\im u} $ 是等距, 而 $ \ker\Star{u}=(\im u)^\bot $, 于是 $ \Star{u} $ 是支撑为 $ \im u $ 的部分等距.

        (2) 因对任意的 $ \xi\in E^\bot $ 总有 $ u\xi=0 $, 因此
        \[
            \forall\xi\in E\,\forall\eta\in H\,(\lrangle{\Star{u}u\xi,\eta}=\lrangle{u\xi,u\eta}=\lrangle{\xi,\eta}),
        \]
        即
        \[
            \forall\xi\in E\,(\Star{u}u\xi=\xi)\land\forall\xi\in E^\bot\,(\Star{u}u\xi=0)
        \]
        于是 $ \Star{u}u=p_E $, 交换 $ u $ 和 $ \Star{u} $ 即可类似证明 $ u\Star{u}=p_{\im u} $.
        
        (3) 由(1), (2) 的证明可得.\qed
    \end{Proof}

    \begin{Theorem}[极分解定理]\index{J!极分解}
        设 $ x\in\CB(H) $, 则 $ x $ 可以被分解成一个部分等距 $ u $ 和一个正算子 $ \abs{x} $ 的乘积, 即
        \[
            x=u\abs{x},\quad \abs{x}=\Star{u}x,
        \]
        其中 $ u $ 的支撑为 $ (\ker x)^\bot $ 且值域为 $ \baro{\im x} $, $ \abs{x} $ 是满足 $ \abs{x}^2=\Star{x}x $ 的正算子, 分解 $ x=u\abs{x} $ 称为算子 $ x $ 的\textbf{极分解}, 且 $ (u,\abs{x}) $ 唯一.
    \end{Theorem}
    \begin{Proof}
        由 $ \forall\xi\in H $ 都有
        \[
            \norm{x\xi}^2=\lrangle{x\xi,x\xi}=\lrangle{\Star{x}x\xi,\xi}=\lrangle{\abs{x}^2\xi,\xi}=\lrangle{\abs{x}\xi,\abs{x}\xi}=\norm{\abs{x}\xi}^2
        \]
        可知 $ \ker x=\ker\abs{x} $, 于是由同态基本定理可知存在唯一的等距 $ u $ 使得下图交换
        \begin{center}
            \begin{tikzpicture}
                \node (A) at (0,0) {$ H $};
                \node (B) at (2.2,0) {$ \dfrac{H}{\im\abs{x}} $};
                \node (C) at (3,0) {$ \oplus $};
                \node (D) at (4,0) {$ (\im\abs{x})^\bot $};
                \node (E) at (0,-1.8) {$ H $};
                \draw[->] (A) -- node[above] {$ \abs{x} $} (B);
                \draw[->] (A) -- node[left] {$ x $} (E);
                \draw[->] (B) -- node[right] {$ u $} (E);
            \end{tikzpicture}
        \end{center}
        从而 $ u $ 延拓到 $ \baro{\im\abs{x}} $ 上仍可成为等距, 再定义 $ u|_{\ker x}=0 $ 即成为 $ H $ 上支撑为 $ \baro{\im \abs{x}}=(\ker x)^\bot $ 的部分等距, 其良定性由同态基本定理保证. 由构造可知 $ \im x\subset\im u\subset\baro{\im x} $, 而部分等距的值域是闭的, 从而 $ \im u=\baro{\im x} $ 满足题设条件.

        再设另有 $ v,y $ 满足题设条件, 则 $ \Star{v}vy=y $, 因此有
        \[
            \Star{x}x=\Star{y}\Star{v}vy=\Star{y}y=y^2,
        \]
        而 $ y $ 是正算子, 从而 $ y=(y^2)^{1/2}=(\Star{x}x)^{1/2}=\abs{x} $. 于是对于任意的 $ \xi\in H $, 都有
        \[
            v\abs{x}\xi=vy\xi=x\xi=u\abs{x}\xi,
        \]
        从而 $ \im\abs{x} $ 上 $ v=u $, 由连续性可延拓到 $ \baro{\im\abs{x}} $ 上.\qed
    \end{Proof}
	
\section*{本章习题}
	\addcontentsline{toc}{section}{本章习题}
	
	习题后面括号中的序号表示对应书中习题的编号.
	
	\begin{enumerate}[label=\textbf{\arabic*.}, ref=\arabic*]
	\item (11.1) 设 $ E $ 是 Banach 空间, $ T\in\CB(E) $. 并设 $ (\lambda_{n})_{n\geqslant1} $ 是 $ \rho(T) $ 中收敛到 $ \lambda\in\K $ 的数列. 证明: 若 $ (R(\lambda_{n}, T)) $ 在 $ \CB(E) $ 中有界, 则 $ \lambda\in\rho(T) $.
	\item (11.3) 设 $ 1\leqslant p\leqslant\infty $ 定义 $ \ell_{p} $ 上的算子 $ S $ (前移算子) 为 $ S(x)(n)=x(n+1) $, 这里 $ x=(x(n))_{n\geqslant1}\in\ell_{p} $.
		\begin{enumerate}[(1)]
			\item 证明: 当 $ p<\infty $ 时, $ \sigma_{p}(S)=\{ \lambda\in\K:\abs{\lambda}<1 \} $; 当 $ p=\infty $ 时, $ \sigma_{p}(S)=\{ \lambda\in\K:\abs{\lambda}\leqslant1 \} $.
			\item 由此导出 $ \sigma(S)=\{ \lambda\in\K:\abs{\lambda}\leqslant1 \} $.
		\end{enumerate}
	\item (11.8) 设 $ E $ 和 $ F $ 是赋范空间, 证明下面的命题成立:
		\begin{enumerate}[(1)]
			\item 若 $ (x_{n})_{n\geqslant1} $ 是 $ E $ 中弱收敛的序列, 则 $ (x_{n})_{n\geqslant1} $ 有界.
			\item 若 $ T\in\CB(E, F) $ 且 $ x_{n} $ 弱收敛到 $ x $, 则 $ T(x_{n}) $ 弱收敛到 $ T(x) $.
			\item 若 $ T\in\CB(E, F) $ 是紧算子且 $ x_{n} $ 弱收敛到 $ x $, 则 $ T(x_{n}) $ 依范数收敛到 $ T(x) $.
			\item 若$ E $自反, $ T\in\CB(E,F) $且当$ x_n $弱收敛到$ x $时, 有$ T(x_n) $依范数收敛到$ T(x) $, 则$ T $是紧算子. 
			\item 若 $ E $ 自反, $ T\in\CB(E, \ell_{1}) $ 或 $ T\in\CB(c_{0}, E) $, 则 $ T $ 是紧算子.
		\end{enumerate}
	\end{enumerate}